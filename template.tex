\documentclass[11pt,a4paper]{article}

%
% This file uses the ethuebung LaTeX package for ETH exercise sheets, which is
% located and maintained at
%
%   https://github.com/phfaist/ethuebung
%
% For more info, please refer to the quick guide or the user's manual which is
% located there (see README.txt).
%
% Please feel free to contact me for any comments, general feedback, questions,
% requests or just to say hi at pfaist@ethz.ch.
%

% -------------------------------------------------------------------------
% Load the ETHUEBUNG package.
% -------------------------------------------------------------------------

%
% NOTE: The same LaTeX file will compile both the exercise sheet and the solutions
% sheet. To select which sheet to generate, uncomment the relevant line below.
% Alternatively, you may use the scripts `pdflatexex' and `pdflatexsol', which will
% also generate a relevant file name ending in "_ex.pdf" or "_sol.pdf" (but for that,
% don't use [sol] below).
%
\usepackage{ethuebung} % comment this and uncomment the next line for solutions
%\usepackage[sol]{ethuebung} % uncomment for solutions


%
% Include other packages that you would wish here.
%
% NOTE: you do NOT have to include the packages `babel', `geometry', `graphicx'
% `amssymb', `amsmath', `enumitem', `ifpdf' or `pdfpages'; they are automatically included
% by default. If you have problems with these packages, such as conflicts with other
% packages or with their options, there exist package options for ethuebung that disable
% loading of these packages; refer to the User's Manual for more details or just contact
% me.
%
\usepackage[utf8]{inputenc}
\usepackage{nicefrac}


% -------------------------------------------------------------------------
% Set up the exercise sheet: Lecture, Semester, Sheet Number, etc.
% -------------------------------------------------------------------------

%
% Sheet language. Defaults to English. Specify here `Deutsch' or `English'. This
% command will also automatically load the `babel' package for that
% language. Alternatively, you may also use the [deutsch] package option (i.e.
% \usepackage[deutsch]{ethuebung}), which has exactly the same effect as the line
% below.
%
% [TIP: when changing language, if you get an error, try to insist by re-running latex,
% or remove the AUX file.]
%
%\UebungLanguage{Deutsch}

%
% Lecture title.
%
\UebungLecture{Quantum Mechanics II.}

%
% Professors giving the lecture. Use several \UebungProf{} commands for multiple
% professors.
%
\UebungProf{Prof. Renato Renner}
%\UebungProf{Prof. Second Professor}

%
% The Semester this lecture is given in.
%
\UebungSemester{FS 2012}

%
% optional: Abgabe date
%
%\UebungDueBy{before the next earthquake}

%
% if you don't like the default title "Series" ...
%
%\UebungsblattTitleSeries{Exercise Sheet}

%
% If you want a special title without numbering, e.g. "Midterm Exam", use the following
% syntax:
%
%\UebungsblattTitleSeries{Midterm Exam.} % Title of what is handed out to students
%\UebungsblattTitleSolutions{Midterm Exam: Solutions.} % The corresponding solutions sheet
%\renewcommand{\uebSerieTitle}{\uebUebungsBlattTitle} % Remove numbering from title

%
% Change the sheet appearance in some way, e.g., you prefer large solutions font; see the
% user's manual for possible styles.
%
%\UebungStyle{LargeSolutions}
%\UebungStyle{ETHUniZH} \ETHUNIBesprechung{that day}{another day}


%
% This usually increases by one every week ...
%
\UebungsblattNumber{10}



% -------------------------------------------------------------------------
% Additional definitions for this sheet: edit & change at wish
% -------------------------------------------------------------------------

\newcommand{\ee}{{\mathrm{e}}}
\newcommand{\abs}[1]{\lvert#1\rvert}
\newcommand{\xabs}[1]{\left\lvert#1\right\rvert}
\let\ReOrig\Re % the old \Re is still available as \ReOrig
\DeclareMathOperator{\Realpart}{Re}
\renewcommand{\Re}{\Realpart}% \Re now gives "Re" instead of a gothic "R"
\newcommand{\ket}[1]{\lvert#1\rangle}
\newcommand{\bra}[1]{\langle#1\vert}
\newcommand{\braket}[2]{\langle#1\hspace{0.25mm}|\hspace{0.25mm}#2\rangle}
\newcommand{\matrixel}[3]{\langle\hspace{0.15mm}#1\hspace{0.25mm}\vert\hspace{0.25mm} #2 \hspace{0.25mm}\vert\hspace{0.25mm}#3\hspace{0.15mm}\rangle}

\newcommand{\xdd}[2]{\frac{\partial #1}{\partial #2}}
\newcommand{\dd}[1]{\frac{\partial}{\partial #1}}
\newcommand{\ddd}[1]{\frac{\partial^2}{\partial #1^2}}


% yep I want some bold vectors...
\newcommand{\boldx}{{\boldsymbol x}}
\newcommand{\boldy}{{\boldsymbol y}}
\newcommand{\boldr}{{\boldsymbol r}}
\newcommand{\boldk}{{\boldsymbol k}}
\newcommand{\boldq}{{\boldsymbol q}}
\newcommand{\nodagger}{{\vphantom{\dagger}}} % no-op, but aligns subscript like the operators that have a dagger



% -------------------------------------------------------------------------
% Now, let's get on to serious stuff.
% -------------------------------------------------------------------------

\begin{document}
%
% Generate the exercise sheet header. If you wish to change the exercise sheet
% appearance, have a look the User's Manual and place the relevant
% \UebungStyle{} or \Uebung***{} commands above in the preamble.
%
\MakeUebungHeader



% -------------------------------------------------------------------------
% Exercise 1: Loosely adapted from Blatter QM2 SS06, Series 10 Ex 1
% -------------------------------------------------------------------------
% note: \uebung{} and \exercise{} are exactly the same command. use one or the other at will.
\exercise{Free Electron Gas.}
% Providing keywords is important, as it will ease future searches within the exercise database:
\keywords{electron gas, Fermi, free electrons, Fermi level, second quantization,
  annihilation operator, creation operator, Bogoliubov transformation, commutation
  relations}


The Hamiltonian of a gas of $N$ free electrons is written in the second quantization formalism as
\begin{align}
  \label{eq:HamiltonianFreeElectronGas}
  H = \sum_{\boldk,s} \xi_k\, c_{\boldk,s}^\dagger c^\nodagger_{\boldk,s}\ ,
\end{align}
where $c_{\boldk,s}$ (resp. $c_{\boldk,s}^\dagger$) is the annihilation (resp. creation) operator of the
electron mode $\boldk,s$ of energy $\xi_k = \epsilon_k - \mu$. (Here $\varepsilon_k=\hbar^2 k^2/(2m)$ and
$\mu$ is the chemical potential, $\mu=E_F$ at $T=0$.) The index $s$ distinguishes the two spin components.

Let's look at excitations that are holes under the Fermi level and electrons above the Fermi level. We
would like to rewrite the Hamiltonian in a form which involves explicitly only these excitations.
We define the creation and annihilation operators of an excitation $\alpha^\dagger_{\boldk,s}$,
$\alpha_{\boldk,s}$ by
\begin{align}
  \label{eq:Alphas}
  \alpha_{\boldk,\uparrow} =
  \begin{cases} c_{\boldk,\uparrow} & \text{for $k>k_F$} \\
    c_{-\boldk,\downarrow}^\dagger & \text{for $k<k_F$} \end{cases}
  \qquad ; \qquad
  \alpha_{\boldk,\downarrow} =
  \begin{cases} c_{\boldk,\downarrow} & \text{for $k>k_F$} \\
    c_{-\boldk,\uparrow}^\dagger & \text{for $k<k_F$} \end{cases}
  \quad .
\end{align}

%
% \begin{exenumerate} ... \end{exenumerate}  starts a list of points to solve, labelled as
% (a), (b), etc.  It can be interrupted, and numbering will resume within the same
% exercise. To change the label, use e.g. \UebungLabelEnum{\alph*)}. You can use \label{}
% and \ref{} as usual. In nested environments you get the inner labelling (i), (ii), ...
% Refer to the user's manual for more information.
%
\begin{exenumerate}
\item Show that the $\alpha$, $\alpha^\dagger$'s obey fermionic commutation relations.
  
  %
  % Use \hint{} or \hints{} to display little hints in your exercise. You can also use
  % \hinweis{} and \hinweise{}, they're aliases of the other commands.
  % 
  \hint{Remember that fermionic creation and annihilation operators $\alpha^\dagger$,
    $\alpha$ obey 
    \begin{align}
      \{\alpha^\nodagger_{\boldk,s}, \alpha^\nodagger_{\boldk',s'}\}
      = \{\alpha_{\boldk,s}^\dagger, \alpha_{\boldk',s'}^\dagger\} = 0
      \quad;\quad
      \{\alpha^\nodagger_{\boldk,s}, \alpha_{\boldk',s'}^\dagger\}
      = \delta_{\boldk,\boldk'}\delta_{s,s'}\ .
    \end{align}
  }

\item Argue that eq.~\eqref{eq:Alphas} is a unitary transformation of the creation and annihilation
  operators. Such a transformation is also called a {\em Bogoliubov transformation}.
  What happens if you act with the annihilators $c_{\boldk,s}$ and
  $\alpha_{\boldk,s}$ on the ground state of the gas?
  
\item Rewrite the Hamiltonian~\eqref{eq:HamiltonianFreeElectronGas} in the form
  \begin{align}
    \label{eq:HamiltonianFreeElGasRewritten}
    H = \sum_\boldk \abs{\xi_k} \left( \alpha_{\boldk\uparrow}^\dagger\alpha_{\boldk\uparrow}
      + \alpha_{\boldk\downarrow}^\dagger\alpha_{\boldk\downarrow} \right) + E_G\quad;\quad
    E_G = 2\sum_{k<k_F} \xi_k\ .
  \end{align}
\end{exenumerate}


%
% NOTE: If you have handwritten scanned solutions in a PDF file, you can attach them to
% this solution sheet with \pdfloesung. This will append the pages present that file to
% this solutions sheet, while placing at this point a reference to see those pages (e.g.
% "For the solution to this exercise, see p. XYZ").
%
%\pdfloesung{the-pdf-file}


%
% Start the solution block. This part will be hidden automatically when generating the
% exercise sheet and will appear only on the solutions sheet.
%
% If you prefer a large solutions font, use \UebungStyle{LargeSolutions} at the beginning
% (in the preamble) of this LaTeX file.
%
% NOTE: Equations in the solutions are labelled as (S.#) (or (L.#) in German), so that you
% can consistently refer to them with \label{} and \eqref{} (or \ref{}) without any
% worries.
%
% The loesung block can be included wherever you want in the exercise: e.g. within each
% \item above or in one block as here. (See exercise 2 for an example.)
%
\begin{loesung}
  \begin{exenumerate}
  \item First assume $k<k_F<k'$. Then
    \begin{align}
      \{\alpha^\nodagger_{\boldk,s}, \alpha^\nodagger_{\boldk',s'}\}
      = \{\alpha_{\boldk,s}^\dagger, \alpha_{\boldk',s'}^\dagger\}
      = \{\alpha^\nodagger_{\boldk,s}, \alpha_{\boldk',s'}^\dagger\} = 0
    \end{align}
    simply because the expressions of $\alpha, \alpha^\dagger$ only involve $c,c^\dagger$'s with the same value
    of $k$ (resp. $k'$), and the $c,c^\dagger$'s involved anticommute because $k\neq k'$.

    Similarly, if $k,k'>k_F$, then the $\alpha,\alpha^\dagger$'s are exactly equal to the $c,c^\dagger$'s and
    the anticommutation relations hold.

    If $k,k'<k_F$, then $\alpha_{\boldk,s} = c_{-\boldk,-s}^\dagger$ and thus
    \begin{align*}
      &\{\alpha_{\boldk,s}, \alpha_{\boldk',s'}\} = \{ c_{-\boldk,-s}^\dagger, c_{-\boldk',-s'}^\dagger\} = 0
      \quad;\quad
      \{\alpha_{\boldk,s}^\dagger, \alpha_{\boldk',s'}^\dagger\} = \{ c_{-\boldk,-s}, c_{-\boldk',-s'}\} = 0
      \quad;\\
      &\{\alpha_{\boldk,s}, \alpha_{\boldk',s'}^\dagger\} = \{ c_{-\boldk,-s}^\dagger, c_{-\boldk',-s'} \}
       =  \{ c_{-\boldk',-s'}, c_{-\boldk,-s}^\dagger \} = \delta_{\boldk,\boldk'}\delta_{s,s'}\ .
    \end{align*}

  \item Eq.~\eqref{eq:Alphas} may be rewritten as
    \begin{align}
      \alpha_{\boldk,\uparrow} =
      \begin{cases} c_{\boldk,\uparrow} & \text{for $k>k_F$} \\
        c_{-\boldk,\downarrow}^\dagger & \text{for $k<k_F$} \end{cases}
      \qquad ; \qquad
      \alpha_{-\boldk,\downarrow}^\dagger =
      \begin{cases} c^\dagger_{-\boldk,\downarrow} & \text{for $k>k_F$} \\
        c_{\boldk,\uparrow} & \text{for $k<k_F$} \end{cases}
      \quad ,
    \end{align}
    which in turn may be expressed as the Bogoliubov transformation
    \begin{align}
      \begin{pmatrix} c_{\boldk\uparrow} \\ c_{-\boldk\downarrow}^\dagger \end{pmatrix}
      = U_\boldk\,
      \begin{pmatrix} \alpha_{\boldk\uparrow} \\ \alpha_{-\boldk\downarrow}^\dagger \end{pmatrix}\ ,
    \end{align}
    where $U_\boldk$ is a $2\times2$ unitary matrix defined as follows:
    \begin{align}
      U_\boldk = \begin{pmatrix} 1 & 0\\ 0 & 1\end{pmatrix}\text{ if $k>k_F$;} \qquad\qquad
      U_\boldk = \begin{pmatrix} 0 & -1\\ 1 & 0\end{pmatrix}\text{ if $k<k_F$.}
    \end{align}

    Note that the combination of operators with reversal of $\boldk$ and $s$ comes from the study of
    Cooper pairs in superconductivity.

    Acting on the ground state with $c_{\boldk,s}$ yields zero for $k>k_F$ but annihilates an electron
    for $k<k_F$. However, acting with $\alpha_{\boldk,s}$ yields zero for any $\boldk,s$ because for $k<k_F$
    $\alpha_{\boldk,s}$ is actually a creation operator $c^\dagger$ which gives zero on an already occupied
    state. This can also be understood as $\alpha$ annihilating excitations: for $k>k_F$, excitations are
    electrons, and in the ground state there are no such electrons to annihilate, and for $k<k_F$ excitations
    are holes, yet in the ground state there are no holes under the Fermi level. So $\alpha$ corresponds
    to an annihilation operator that yields zero when acting on the ground state, this is what one usually
    wants.
    
  \item Eq.~\eqref{eq:HamiltonianFreeElGasRewritten} represents a different way of counting the energy of the
    system. Instead of counting all electrons and their energies, we count the holes for $k<k_F$ and the
    electrons for $k>k_F$. Here it is important that the energies $\xi_k$ are scaled such that at the Fermi level
    $\xi_{k=k_F} = 0$. Note that we could not have just counted the electrons above the Fermi energy and
    multiplied by two, because we wouldn't know by how much energy such an electron would have been excited.
    
    The transformation~\eqref{eq:Alphas} effectively means $c_{\boldk,s} = \alpha_{\boldk,s}$ for $k>k_F$ and
    $c_{\boldk,s} = \alpha^\dagger_{-\boldk,-s}$ for $k<k_F$. Now write
    \begin{align*}
      H ~ = \sum_{k<k_F,\,s} \xi_k\,c_{\boldk,s}^\dagger c_{\boldk,s} &+
      \sum_{k>k_F,\,s} \xi_k\,c_{\boldk,s}^\dagger c_{\boldk,s} \ 
      = \sum_{k<k_F,\,s} \xi_k\,\alpha_{-\boldk,-s}\alpha_{-\boldk,-s}^\dagger +
      \sum_{k>k_F,\,s} \xi_k\alpha_{\boldk,s}^\dagger\alpha_{\boldk,s} \\
      &= \sum_{k<k_F,\,s} \xi_k\left(1-\alpha^\dagger_{-\boldk,-s}\alpha_{-\boldk,-s}\right)
      + \sum_{k>k_F,\,s} \xi_k\,\alpha_{\boldk,s}^\dagger \alpha_{\boldk,s} \\
      &= 2\sum_{k<k_F} \xi_k  +  \sum_{k<k_F,\,s} \abs{\xi_k}\,\alpha_{\boldk,s}^\dagger\alpha_{\boldk,s}
      + \sum_{k>k_F,\,s} \abs{\xi_k}\,\alpha^\dagger_{\boldk,s}\alpha_{\boldk,s}\ ,
    \end{align*}
    where we have anticommutated $\alpha_{-\boldk,-s}$ with $\alpha^\dagger_{-\boldk,-s}$, % (normal ordering),
    cancelled the
    minus sign with the sign of $\xi_k$ (remember: $\xi_k < 0$ for $k<k_F$ and $\xi_k > 0$ for $k>k_F$),
    relabeled $\boldk\rightarrow-\boldk$, $s\rightarrow -s$ in the second sum on the last line, and obtained a
    factor $2$ in the first term from the summation over the index $s$.

    Eventually,
    \begin{align}
      H = 2\sum_{k<k_F} \xi_{k} + \sum_{\boldk,s} \,\abs{\xi_k}\,\alpha^\dagger_{\boldk,s}\alpha_{\boldk,s}\ .
    \end{align}
  \end{exenumerate}
\end{loesung}



% -------------------------------------------------------------------------
% Exercise 2: Calculations from Renato's QM2 Script, Section 5.1
% -------------------------------------------------------------------------
\uebung{Correlation Functions in a Fermi Sea.}
\keywords{correlation function, fermi sea, second quantization, electron gas, fermions,
  quantum fields, one-particle correlation function, pair correlation function, momentum
  modes, annihilation operator, creation operator}


Consider a gas of $N$ identical fermions with spin $\nicefrac{1}{2}$. The fermions are free and non-interacting.
The ground state is then given by
\begin{align}
  \ket{\Phi_0} = \prod_{\abs\boldk\leqslant k_F,\,s}\!\!\!\! a_{\boldk\,s}^\dagger\:\ \ket0\ .
\end{align}
One defines the {\em one-particle correlation function} $G_s(\boldx-\boldy)$ as
\begin{align}
  \label{eq:TwoPointCorrFunc}
  G_s(\boldx-\boldy) = \frac{n}{2}\,g_s(\boldx-\boldy)
  = \matrixel{\Phi_0}{\Psi_s^\dagger(\boldx)\Psi_s(\boldy)}{\Phi_0}\ .
\end{align}
This is the amplitude of recreating a fermion of spin $s$ at position $\boldx$ when one was annihilated
at position $\boldy$ with same spin.

\begin{exenumerate}
\item Using explicit expressions for the field operators $\Psi_s(\boldx)$, calculate $G_s(\boldx-\boldy)$ and
  sketch its graph as function of $\abs{\boldx-\boldy}$.
  Show that $\lim_{\boldr\rightarrow 0} G_s(\boldr) = \frac{n}{2}$
  and $\lim_{r\rightarrow \infty} G_s(\boldr) = 0$.
  
  % 
  % See, you can place \begin{loesung}...\end{loesung} within the exercise if you prefer.
  % 
  \begin{loesung}
    Recall the definition of the field operators $\Psi_s(\boldx)$,
    \begin{align}
      \Psi_s(\boldx) = \sum_{\text{modes}} \phi_i(\boldx)\,a_i\ .
    \end{align}
    For free fermions we choose to consider the momentum modes designated by $\boldk$, for which the wave
    functions are plane waves,
    \begin{align}
      \phi_\boldk(\boldx) = \frac{1}{\sqrt V}\,\ee^{i\boldk\cdot\boldx}\ .
    \end{align}

    Now insert the expression for the $\Psi$'s into~\eqref{eq:TwoPointCorrFunc},
    \begin{align}
      \frac{n}{2}\,g_s(\boldx-\boldy)
      &= \frac{1}{V}\,\sum_{\boldk_1\,\boldk_2} \ee^{-i\boldk_1\boldx + i\boldk_2\boldy} \,
      \matrixel{\Phi_0}{a^\dagger_{\boldk_1,s} a^{\vphantom{\dagger}}_{\boldk_2,s}}{\Phi_0}
      = \frac{1}{V}\,\sum_{\boldk_1\,\boldk_2} \ee^{-i\boldk_1\boldx + i\boldk_2\boldy} \,
      \matrixel{\Phi_0}{n_{\boldk_1,s}}{\Phi_0}\,\delta_{\boldk_1\boldk_2}\nonumber\\
      &= \frac{1}{V}\,\sum_{\boldk} \ee^{-i\boldk\left(\boldx-\boldy\right)} \,
      \matrixel{\Phi_0}{n_{\boldk,s}}{\Phi_0}
      = \frac{1}{V} \sum_{k\leqslant k_F} \ee^{-i\boldk\cdot\left(\boldx-\boldy\right)}
      \label{eq:calcgsxy1}
    \end{align}
    Indeed, because $\Phi_0$ is a number basis element, any annihilated fermion has to be recreated to give a nonzero
    matrix element; in the ground state, all levels under the Fermi level are occupied, such that
    $\matrixel{\Phi_0}{n_{\boldk,s}}{\Phi_0}=1$ for $k<k_F$ and $\matrixel{\Phi_0}{n_{\boldk,s}}{\Phi_0}=0$
    for $k>k_F$. Pursuing the calculation,
    \begin{align}
      \text{\eqref{eq:calcgsxy1}} &= \frac{1}{V} \sum_{k\leqslant k_F} \ee^{-i\boldk\cdot\left(\boldx-\boldy\right)}
      \approx
      \int_{k\leqslant k_F} \frac{\mathrm{d}^3\boldk}{(2\pi)^3}\,\ee^{-i\boldk\cdot\left(\boldx-\boldy\right)}
      = \frac{1}{(2\pi)^3} \int_0^{2\pi}\!\!\mathrm d\phi\int_0^{k_F}\!\! \mathrm d k\, k^2
      \int_{-1}^{+1}\! \mathrm d\cos\theta\,\,\ee^{i k \abs{\boldx-\boldy}\cos\theta} \nonumber\\
      &= \frac{1}{(2\pi)^2} \int_0^{k_F} \mathrm d k\,k^2 \frac1{i\,k\,\abs{\boldx-\boldy}}
      \,2i\sin\left(k\abs{\boldx-\boldy}\right)
      = \frac{1}{2\pi^2\,\abs{\boldx-\boldy}} \int_0^{k_F} \mathrm d k\,k \sin\left(k\abs{\boldx-\boldy}\right)
      \nonumber\\
      &= \frac{n}{2}\cdot 3\cdot \left.\frac{\sin x - x\cos x}{x^3}\right\rvert_{x=k_F\abs{\boldx-\boldy}}\ .
    \end{align}
    In the last step one calculates using elementary analysis
    \begin{align}
      \int_0^{k_F}\!\mathrm d k\,k\sin\left(k\abs{\boldx-\boldy}\right)
      = \frac{1}{\abs{\boldx-\boldy}^2}\Bigl[\,\sin\left(k_F \abs{\boldx-\boldy}\right)
        - k_F \abs{\boldx-\boldy} \cos\left(k_F\abs{\boldx-\boldy}\right) \,\Bigr]\ ,
    \end{align}
    and applies the relation $k_F=\left[3\pi^2\,n\right]^{1/3}$ (Eq.~(5.1.4) in the lecture notes).

    We have replaced the sum by an integral in the previous calculation by sending $V\rightarrow\infty$. There
    the distribution
    $\frac{1}{V}\,\ee^{-i\boldk\cdot\left(\boldx-\boldy\right)}$ is replaced by
    $\frac{1}{(2\pi)^3}\,\ee^{-i\boldk\cdot\left(\boldx-\boldy\right)}$.

    If $\boldr\rightarrow 0$, then
    \begin{align}
      \frac{n}{2}\,g_s\left(\boldr\right)
      &= \frac{n}{2}\cdot 3\cdot
      \left.\frac{x - \frac{1}{3!}\,x^3 + O(x^5) - x + \frac12\,x^3 + O(x^5)}{x^3}\right\rvert_{x=k_F\abs\boldr}
      = \frac{n}{2}\cdot 3\cdot\left.\frac{\frac{1}{3}\,x^3 + O(x^5)}{x^3}\right\rvert_{x=k_F\abs\boldr}
      \longrightarrow \frac n 2\ .
    \end{align}

    Likewise, if $r\rightarrow\infty$, then
    \begin{align}
      \xabs{\frac n 2 \, g_s\left(\boldr\right)}
      = \frac n 2 \cdot 3 \xabs{ \frac{\sin x}{x^3} - \frac{x\cos x}{x^3} }
      \leqslant \frac n 2 \cdot 3 \left[ \frac1{x^3} + \frac{x}{x^3} \right] \longrightarrow 0 \ .
    \end{align}

    A plot of $g_s\left(\boldr\right)$ is given in the following figure.
    %
    % The easiest way to include a figure inline is to bypass completely
    % \begin{figure}...\end{figure} and use \includegraphics{} directly. You don't get any
    % (sometimes superfluous) caption, and it's guaranteed to be right there where you
    % asked for it. Of course, if you want a caption, or if you want to refer to it as
    % Fig. X.Y, you can use \begin{figure} ... \end{figure} as you usually do. The User's
    % Manual has some notes on this subject, including how to wrap text around your
    % figures.
    % 
    \begin{center}
      \vspace*{0.2cm} % add vertical spacing if needed (negative value = remove spacing)
      %
      % Use \includegraphics{} to actually insert the figure.
      % 
      %\includegraphics[width=8cm]{CorrFuncPlot}
      %
      %
      % Instead, for this example, we'll display some dummy text. (So that you don't need
      % the figure file to compile this template.) Oh, and please don't forget to remove
      % this junk from your final exercise sheet, right?
      %
      \framebox{\begin{minipage}{8cm}\begin{center}\vspace*{2mm}{\Large\bfseries WANTED: FIGURE}\\[3mm](sorry, no picture available)\\[4mm]Cool figure. Looks cool, feels cool, and displays cool stuff. Dead or alive. (Alive is cooler.)\end{center}\end{minipage}}
    \end{center}

  \end{loesung}
\end{exenumerate}

Likewise, one can define the {\em pair correlation function} $g_{ss'}(\boldx-\boldy)$ by
\begin{align}
  \label{eq:Paarkorrelationsfunktion}
  \Bigl(\frac{n}{2}\Bigr)^2 g_{ss'}(\boldx-\boldy)
  = \matrixel{\Phi_0}{\Psi_s^\dagger(\boldx) \Psi_{s'}^\dagger(\boldy) \Psi_{s'}(\boldy)\Psi_s(\boldx)}{\Phi_0}\ .
\end{align}

\begin{exenumerate}
\item Rewrite Eq.~\eqref{eq:Paarkorrelationsfunktion} in the form
  \begin{align}
    \label{eq:PaarkorrelationsfunktionExplicitPsis}
    \Bigl(\frac{n}{2}\Bigr)^2 g_{ss'}(\boldx-\boldy)
    = \frac1{V^2}\sum_{\boldk_1\,\boldk_2\,\boldq_1\,\boldq_2}
    \ee^{-i\left(\boldk_1-\boldk_2\right)\cdot\boldx}\, \ee^{-i\left(\boldq_1-\boldq_2\right)\cdot\boldy}\,
    \matrixel{\Phi_0}{a^\dagger_{\boldk_1,s} a^\dagger_{\boldq_1,s'}
      a^\nodagger_{\boldq_2,s'} a^\nodagger_{\boldk_2,s} }{\Phi_0}\ .
  \end{align}

  \begin{loesung}
    This follows directly by inserting explicit expressions of $\Psi_s(\boldx)$, analogously to point~(a).
  \end{loesung}
  
\item Assume first that $s\neq s'$. Calculate $g_{ss'}(\boldx-\boldy)$.
  
  \begin{loesung}
    Considering~\eqref{eq:PaarkorrelationsfunktionExplicitPsis} and assuming $s\neq s'$, we must
    have $\boldk_1=\boldk_2$
    and $\boldq_1 = \boldq_2$, or the matrix element is zero because we would fail to recreate the particle
    that we annihilated and we would end up with a state orthogonal to the ground state. (This argument holds
    since $\ket{\Phi_0}$ is a number eigenstate.) So
    \begin{align}
      \Bigl(\frac{n}{2}\Bigr)^2 g_{ss'}(\boldx-\boldy)
      = \frac1{V^2}\sum_{\boldk\,\boldq}\matrixel{\Phi_0}{n_{\boldk,s} n_{\boldq,s'}}{\Phi_0}
      = \Bigl(\frac{n}{2}\Bigr)^2\ ,
    \end{align}
    because $\frac{1}{V}\,\sum_{k<k_F}1 = \frac{n}{2}$ (the factor $\frac12$ is due to spin degeneracy).
    Finally
    \begin{align}
      g_{s\neq s'}\left(\boldx-\boldy\right) = 1\ .
    \end{align}
  \end{loesung}
  
\item Now consider the case where $s=s'$ and calculate $g_{ss}(\boldx-\boldy)$. Plot the quantity
  $g_{ss}(\boldx-\boldy)$ as a function of $\abs{\boldx-\boldy}$.

  \begin{loesung}
    If $s=s'$, then we must have in~\eqref{eq:PaarkorrelationsfunktionExplicitPsis} either $\boldk_1=\boldk_2$
    and $\boldq_1=\boldq_2$, or $\boldk_1=\boldq_2$ and $\boldq_1=\boldk_2$. This can be written as
    \begin{align}
      \matrixel{\Phi_0}{a^\dagger_{\boldk_1,s}a^\dagger_{\boldq_1,s}
        a^\nodagger_{\boldq_2,s}a^\nodagger_{\boldk_2,s}}{\Phi_0}
      = \delta_{\boldk_1,\boldk_2}\delta_{\boldq_1,\boldq_2}\,
      \matrixel{\Phi_0}{n_{\boldk_1,s} n_{\boldq_1,s}}{\Phi_0}
      - \delta_{\boldk_1,\boldq_2}\delta_{\boldq_1,\boldk_2}\,
      \matrixel{\Phi_0}{n_{\boldk_1,s} n_{\boldq_1,s}}{\Phi_0}\ ,
    \end{align}
    where the minus sign comes from having anticommuted an odd number of $a$'s to obtain the number operators in
    the second term.

    Recalling Eq.~\eqref{eq:PaarkorrelationsfunktionExplicitPsis}, we have
    \begin{align}
      \Bigl(\frac{n}{2}\Bigr)^2 g_{ss}(\boldx-\boldy)
      &= \frac{1}{V^2} \sum_{\boldk\,\boldq}
      \left[1 - \ee^{-i\left(\boldk-\boldq\right)\cdot\boldx}\ee^{-i\left(\boldq-\boldk\right)\cdot\boldy}\right]
      \matrixel{\Phi_0}{n_{\boldk,s}n_{\boldq,s}}{\Phi_0} \nonumber\\
      &= \frac{1}{V^2} \sum_{\boldk\,\boldq}
      \left[1 - \ee^{-i\left(\boldk-\boldq\right)\cdot\left(\boldx-\boldy\right)}\right]
      \matrixel{\Phi_0}{n_{\boldk,s}n_{\boldq,s}}{\Phi_0} \nonumber\\
      &= \frac{1}{V^2} \left[ \Bigl(\frac{N}{2}\Bigr)^2
        - \sum_{k,q<k_F} \ee^{-i\boldk\left(\boldx-\boldy\right)}\ee^{-i\boldq\left(\boldx-\boldy\right)} \right]
      = \Bigl(\frac{n}{2}\Bigr)^2\,\left[\,1 - g_s(\boldx-\boldy)\,^2\,\right]\ ,
    \end{align}
    where in the last line, the sign in the second exponential was flipped by sending $\boldq\rightarrow-\boldq$,
    and the quantity $g_s(\boldx-\boldy)$ appeared by recognizing the expression in~\eqref{eq:calcgsxy1}.

    Recalling the result for $g_s(\boldx-\boldy)$, known from point (a),
    \begin{align}
      g_{ss}(\boldx-\boldy)
      = 1 - 9\cdot\left.\frac{\left(\sin x - x\cos x\right)^2}{x^6}\right\rvert_{x=k_F\abs{\boldx-\boldy}}\ .
    \end{align}

    A plot of $g_{ss}(\boldx-\boldy)$ can be found in the lecture notes, p.~69, Fig.~10.
  \end{loesung}
\end{exenumerate}



\end{document}
