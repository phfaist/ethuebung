\documentclass[11pt,a4paper]{article}

% ETHUEBUNG package
% Use pdflatexex/pdflatexsol to compile exercise sheet/solutions respectively
\usepackage{ethuebung}

\UebungLecture{Lecture Title.}
\UebungProf{Prof. Zebigboss}
\UebungSemester{HS 2999}

\UebungsblattNumber{8}

% This is to remove the "Solutions." label.
\UebungSolLabel{}

% STYLE: musterloesung with normal solutions font and exercise data in italic
\UebungStyle{LargeSolutions}


\begin{document}
\MakeUebungHeader

% -------------------------------------------------------------------------
%\exercise{} would work just as fine.
\uebung{Getting to Know the Qubit.}

In this exercise, you will be asked to do some work.

\begin{onlymusterloesung}
  This text only appears in the solutions.
  \begin{equation}
    f = \text{suspense}\ldots
  \end{equation}
\end{onlymusterloesung}


% starts a list of points to solve (a), (b), ...
\begin{exenumerate}
\item Stare at this equation for at least half an hour.
  \begin{align}
    A = B.
  \end{align}

  \begin{onlymusterloesung}
    This text only appears in the solutions.
    \begin{equation}
      f = \text{suspense}\ldots
    \end{equation}
  \end{onlymusterloesung}

\item Solve it.
  \hint{You might need to specify the unknown.}

  \begin{loesung}% the solution to this exercise part. \begin{solution} is the same
    The solution is given by ...... 
    \begin{align}
      A=B\ ,
    \end{align}
    of course.

    [Note that the equation in the solution has a separate numbering !]
  \end{loesung}
\end{exenumerate}

There's a second part to this exercise (horrayy!)

\begin{exenumerate} % {exenumerate} environment automatically resumes correct numbering
\item \label{expart:Sleep} Now go to bed and sleep for at least 8 hours.
\item Convince your professor that you did some great work during last night when you did
  point~\ref{expart:Sleep}.
\end{exenumerate}


% -------------------------------------------------------------------------
\uebung{Just a second exercise.}

More stuff...

\begin{solution} % \begin{solution} and \begin{loesung} are the same
  Well there was actually nothing to do.
\end{solution}





\end{document}
