\documentclass[11pt,a4paper]{article}

% ETHUEBUNG package
% Use pdflatexex/pdflatexsol to compile exercise sheet/solutions respectively
\usepackage{ethuebung}

\UebungLecture{Lecture Title.}
\UebungProf{Prof. Zebigboss}
\UebungSemester{HS 2999}

\UebungLabelEnum{(\Alph*)}

\UebungsblattNumber{8}

\begin{document}
\MakeUebungHeader

% -------------------------------------------------------------------------
\uebung{Some Item Back-Referencing.}

This demo is about exercise part numbering and back-referencing.

\begin{exenumerate}
\item \label{expart:A}
  This first item should be labelled ``(A)''.

\item \label{expart:B}
  This second item should be labelled ``(B)'', and appears after item~\ref{expart:A}.

  Here are some claims:
  \begin{exenumerate}
  \item \label{expart:pinotrational} \exenumfulllabel{expart:fullpinotrational}
    $\pi$ is not rational.
  \item \label{expart:Bsqrt2notrational}
    $\sqrt{2}$ is not rational.
  \item \exenumfulllabel{expart:fBprime}
    2350545375697400924 is a prime number.
  \end{exenumerate}

  Prove that the claims~\ref{expart:pinotrational} and~\ref{expart:Bsqrt2notrational} are
  true.

\end{exenumerate}  


Not all claims presented in~\ref{expart:B} are that easy to prove. 

\begin{exenumerate} % note that the numbering resumes automatically at (C)
\item Give an explicit proof for parts~\ref{expart:fullpinotrational}
  and~\ref{expart:fBprime}.

  \begin{solution}
    A proof for $\pi$ not being rational can be found in any standard textbook.

    The given number is not prime, it can be decomposed at least as
    \begin{align}
      2350545375697400924 = 547657356872647 * 4292\ .
    \end{align}
  \end{solution}
\end{exenumerate}



\end{document}

%%% Local Variables: 
%%% mode: latex
%%% TeX-master: t
%%% End: 
