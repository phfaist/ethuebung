\documentclass[11pt,a4paper]{article}

% ETHUEBUNG package
% Use pdflatexex/pdflatexsol to compile exercise sheet/solutions respectively
\usepackage{ethuebung}

\UebungLecture{Lecture Title.}
\UebungProf{Prof. Zebigboss}
\UebungSemester{HS 2999}

\UebungsblattNumber{8}

\begin{document}
\MakeUebungHeader


\begin{tips}
Some recap on blah blah blah blah.
\end{tips}


% -------------------------------------------------------------------------
%\exercise{} would work just as fine.
\uebung{Getting to Know the Qubit.}

In this exercise, you will be asked to do some work.

% starts a list of points to solve (a), (b), ...
\begin{exenumerate}
\item Stare at this equation for at least half an hour.
  \begin{align}
    \label{eq:ToStareAt}
    A = B.
  \end{align}

  \begin{tips}
    You should start by staring on the left hand side of equation~\eqref{eq:ToStareAt} for
    at least 30 min, then do the same on the right hand side of that equation.
  \end{tips}
  
\item Solve it.
  \hint{You might need to specify the unknown.}

  \begin{tips}
    In equation~\eqref{eq:ToStareAt}, there are two symbols, $A$ and $B$. Is it possible
    to solve this equation for both unknowns at the same time? What happens if you fix one
    to some constant value?
  \end{tips}

  \begin{loesung}% the solution to this exercise part. \begin{solution} is the same
    The solution is given by ...... 
    \begin{align}
      A=B\ ,
    \end{align}
    of course.

    [Note that the equation in the solution has a separate numbering !]
  \end{loesung}
\end{exenumerate}

There's a second part to this exercise (horrayy!)

\begin{exenumerate} % {exenumerate} environment automatically resumes correct numbering
\item \label{expart:Sleep} Now go to bed and sleep for at least 8 hours.
\item Convince your professor that you did some great work during last night when you did
  point~\ref{expart:Sleep}.
\end{exenumerate}


% -------------------------------------------------------------------------
\uebung{Just a second exercise.}

More stuff...

\begin{solution} % \begin{solution} and \begin{loesung} are the same
  Well there was actually nothing to do.
\end{solution}





\end{document}
