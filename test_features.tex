\documentclass[11pt,a4paper]{article}

\usepackage{ethuebung} % comment this and uncomment the next line for solutions
%\usepackage[sol]{ethuebung} % uncomment for solutions, or use \UebungMakeSolutionsSheet.

\UebungLanguage{Deutsch}

%\UebungMakeSolutionsSheet % UNCOMMENT for solutions sheet, or use [sol] package option

\UebungLecture{Lecture Title.}
\UebungProf{Prof. Zebigboss}
\UebungSemester{HS 2999}

\UebungStyle{PreviousITP}
\UebungLabelEnum{[\Alph*/\Roman*]}

\UebungsblattNumber{8}

\begin{document}
\MakeUebungHeader

% -------------------------------------------------------------------------
%\exercise{} would work just as fine.
\uebung{Getting to Know the Qubit.}

In this exercise, you will be asked to do some work.


\begin{exenumerate}
\item Stare at this equation for at least half an hour.
  \begin{align}
    A = B.
  \end{align}

\item Solve it.
  \hint{You might need to specify the unknown.}

  \begin{loesung}
    The solution is given by ...... 
    \begin{align}
      A=B\ ,
    \end{align}
    of course.

    [Note that the equation in the solution has a separate numbering !]
  \end{loesung}
\end{exenumerate}

There's a second part to this exercise (horrayy!)

\exenumfulllabel{expart:ThisIsALabelOutsideExenumEnvironment}

\begin{exenumerate} % {exenumerate} environment automatically resumes correct numbering!
\item \exenumfulllabel{expart:FullDoNothing}
  Please do:
  \begin{exenumerate}
  \item \label{expart:Sleep}
    Now go to bed and sleep for at least 8 hours.
  \item \exenumfulllabel{expart:FullSleep}
    Do nothing else.
  \item Do nothing more.
  \end{exenumerate}

\item Convince your professor that you did some great work during last night when you did
  point~\ref{expart:Sleep} and point~\ref{expart:FullSleep}, in
  point~\ref{expart:FullDoNothing}.

\item Write a summary on density operators.
  \pdfloesung{fig/Summary_DensityOperator.pdf}

\end{exenumerate}

This text: `\ref{expart:ThisIsALabelOutsideExenumEnvironment}' should be empty.


\subexercise{\LaTeX{} package.}

\begin{exenumerate}
\item Also, write a small manual on how to use the \texttt{ethuebung} package.
\end{exenumerate}

% Attach the solution to this sub-exercise at the end of the document (this will insert
% the given pdf document at the end of this sheet, with proper references and a title).
% See pdfpages package manual for possible options here (options to \includepdf[]{}).
%   http://mirrors.ctan.org/macros/latex/contrib/pdfpages/pdfpages.pdf
\pdfsolution[pages={1}]{fig/Summary_SymmRotSpin.pdf}


%Please do *not* abuse of the ugly \ifmusterloesung / \ifuebungsblatt construct!
CURRENT SHEET (\texttt{\textbackslash ifmusterloesung}):
\ifmusterloesung MUSTERLOESUNG \else EXERCISE SHEET \fi .\\
CURRENT SHEET (\texttt{\textbackslash ifuebungsblatt}):
\ifuebungsblatt UEBUNGSBLATT \else SOLUTIONS SHEET \fi .

\begin{onlysolutions} % also {onlymusterloesung}
Text that will appear in the solutions sheet only, and that will be hidden if the sheet is
compiled in exercise sheet mode (i.e. not solutions mode).
\end{onlysolutions}
\begin{onlyuebungsblatt} % also {onlyexercisesheet}
Text that will appear in the exercise sheet only, and that will be hidden in solutions
mode. You can use this to insert page break commands in one version of the sheet only:

% force following content on a new page. Useful if you're at the end of an exercise, and
% the following exercise starts two lines from the bottom of the page...
\newpage
\end{onlyuebungsblatt}

Did Babel load correctly?: \texttt{\textbackslash figurename=}\figurename.



% -------------------------------------------------------------------------
\uebung{Just a second exercise.}

More stuff...

\begin{solution}
  Well there was actually nothing to do.
\end{solution}

% change sub-exercise format
\UebungSubLabel{Teil}

\subuebung{Something}
Hi there! Note that this text is on the same line as the exercise title because we have
not left a newline between the \texttt{\textbackslash subuebung} command and the question
text.

\subuebung{Something else}

Hi there! Note that this text starts on a new line.



\end{document}
