\documentclass[11pt,a4paper]{article}

\usepackage[titlesmall]{docnote}
\usepackage[T1]{fontenc}
\usepackage{framed}
\usepackage{verbatim}
\usepackage{docethuebung}
\usepackage{url}

\let\docpkgcmd\pkgcmd
\renewcommand{\pkgcmd}[1]{\docpkgcmd[]{#1}}
\let\docpkgenv\pkgenv
\renewcommand{\pkgenv}[1]{\docpkgenv[]{#1}}

\setlength\parindent{0pt}

\title{Quick Guide to Package {\fontseries{m}\selectfont ethuebung}}
\author{Philippe Faist, \texttt{pfaist@ethz.ch}}
\date\today

\begin{document}
\maketitle

This document provides a quick intro to the usage of the \texttt{ethuebung} package for
exercise sheets. Please refer to the complete documentation for a more complete overview
of the features of this package.

The philosophy of this package is to have all contents related to an exercise contained in
a \emph{single} \LaTeX{} file, which then can generate both the exercise sheet and the
solutions sheet.


\section*{Getting Started.}

Simply copy the style file \texttt{ethuebung.sty} into the same directory as your exercise
sheet \LaTeX{} file. No other files are needed (no logo images etc. are needed as they are
packaged with the style file).
Then start off with the following template.

\begin{pkgverbatim}%
\verbatiminput{verbatim_minimal_template.tex}%
\end{pkgverbatim}

\section*{Setting up the Exercise Sheet.}

Obviously you'll have to fill in the title of your lecture, the lecturer and the semester
accordingly using the \pkgcmd{UebungLecture}, \pkgcmd{UebungProf} or
\pkgcmd{UebungLecturer}, and \pkgcmd{UebungSemester} commands. Use the
\pkgcmd{UebungsblattNumber} command to specify the number of the exercise sheet (usually
increases by one every week). Don't forget the \pkgcmd{MakeUebungHeader} command at the
beginning of the document to actually draw the header.


\section*{The Solutions Sheet.}

Write the solutions to each exercise inline, using a \pkgenv{loesung} or \pkgenv{solution}
environment. When you compile your file (as in the template), then the solutions will
simply be discarded and will not appear in the document.

If you want to generate a solutions sheet with the solutions displayed (eg. for the TA's
or for the students after they handed in), then simply change the two lines:
\begin{pkgverbatim}
\begin{verbatim}
\usepackage{ethuebung} % comment this and uncomment next line for solutions
%\usepackage[sol]{ethuebung} % uncomment for solutions
\end{verbatim}
\end{pkgverbatim}
to:
\begin{pkgverbatim}
\begin{verbatim}
%\usepackage{ethuebung} % comment this and uncomment next line for solutions
\usepackage[sol]{ethuebung} % uncomment for solutions
\end{verbatim}
\end{pkgverbatim}

Recompile the sheet, and the solutions will be displayed.


\section*{Language Selection.}

If you want the {\bfseries German} version of the template, simply pass the
\texttt{[deutsch]} option to the \cmdname{usepackage} directive,
\begin{pkgverbatim}
\begin{verbatim}
  \usepackage[deutsch]{ethuebung} % German version, "Uebungsblatt"
  %\usepackage[deutsch,sol]{ethuebung} % German verion, "Musterloesung"
\end{verbatim}
\end{pkgverbatim}


\section*{Teaser---Some More Features}

Here are some features implemented in this package. If you want to use them, refer to the
main documentation.

\begin{itemize}
\item The \pkgenv{exenumerate} environment behaves very much like the \pkgenv{enumerate}
  environment, with automatic (a), (b), ... numbering (and (i), (ii) nested numbering),
  and more features.
\item All aspects of the exercise sheet are customizable without modifying the style
  file. Check out the documentation.
\item You can use different styles to change the appearance of your sheet---check out the
  \pkgcmd{UebungStyle} command.
\item If you have e.g. scanned handwritten solutions, you can attach them at the end of
  the solutions sheet automatically. Use the \pkgcmd{pdfloesung} command.
\item ...
\end{itemize}


\end{document}

%%% Local Variables: 
%%% mode: latex
%%% TeX-master: t
%%% End: 
