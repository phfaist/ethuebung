\documentclass[11pt,a4paper]{article}

\usepackage{docnote}
\usepackage[T1]{fontenc}
\usepackage{framed}
\usepackage{verbatim}
\usepackage{docethuebung}


\title{Package {\fontseries{m}\selectfont ethuebung} for ETH Exercise Sheets}
\author{Philippe Faist, \texttt{pfaist@ethz.ch}}
\date\today

\begin{document}
\maketitle

\renewcommand{\abstractname}{}
\setlength{\noteabstracttextwidth}{0.95\textwidth}
\renewcommand{\noteabstracttextfont}{\small}
\begin{abstract}
This package provides a unified way of typing exercises for ETH Zurich. While you type in
logically all aspects of your exercise using provided \LaTeX{} macros (title, text, hints,
solution, etc.), it is rendered according to some standard style (yet remaining highly
customizable), and provides different versions of the sheet for distributing to students
(without the solutions), or for TA's (with solutions).
\end{abstract}

\inlinetoc

\section{Quick Start Guide and Simple Usage.}
\label{sec:quickstart}

Copy the style file \texttt{ethuebung.sty} into the same directory as your exercise sheet
\LaTeX{} file. No other files are needed (no logo images etc. are needed as they are
packaged with the style file). Start off with the following minimal template.

\begin{pkgverbatim}%
\verbatiminput{verbatim_minimal_template.tex}%
\end{pkgverbatim}

If you want the {\bfseries German} version of the template, simply pass the
\texttt{[german]} option to the \cmdname{usepackage} directive (see also
Sec.~\ref{sec:Language}),
\begin{pkgverbatim}
\begin{verbatim}
  \usepackage[deutsch]{ethuebung} % German version, "Uebungsblatt"
  %\usepackage[deutsch,sol]{ethuebung} % German verion, "Musterloesung"
\end{verbatim}
\end{pkgverbatim}

Obviously you'll have to fill in the title of your lecture, the lecturer and the semester
accordingly using the \pkgcmd{UebungLecture}, \pkgcmd{UebungProf} or
\pkgcmd{UebungLecturer}, and \pkgcmd{UebungSemester} commands. Use the
\pkgcmd{UebungsblattNumber} command to specify the number of the exercise sheet (usually
increases by one every week). Don't forget the \pkgcmd{MakeUebungHeader} command at the
beginning of the document to actually draw the header.

Write the solutions to each exercise inline, using a \pkgenv{loesung} or \pkgenv{solution}
environment. When you compile the code above, then the solutions will not appear.

If you want to generate a solutions sheet with the solutions displayed (eg. for the TA's
or for the students after they handed in), then simply change the two lines:
\begin{pkgverbatim}
\begin{verbatim}
\usepackage{ethuebung} % comment this and uncomment next line for solutions
%\usepackage[sol]{ethuebung} % uncomment for solutions
\end{verbatim}
\end{pkgverbatim}
to:
\begin{pkgverbatim}
\begin{verbatim}
%\usepackage{ethuebung} % comment this and uncomment next line for solutions
\usepackage[sol]{ethuebung} % uncomment for solutions
\end{verbatim}
\end{pkgverbatim}

Recompile the sheet, and the solutions will be displayed.


\section{What This Package Does}

This package provides a unified way of typing exercises for ETH Zurich. While you type in
logically all aspects of your exercise using provided \LaTeX{} macros (title, text, hints,
solution, etc.), it is rendered according to some standard style (yet remaining highly
customizable), and provides different versions of the sheet for distributing to students
(without the solutions), or for TA's (with solutions).



ex numbering

both ex sheet/solution sheet

  - eqns in solution numbered by themselves
  - can attach pdfs for solution


\section{Setting Up The Exercise Sheet}
\label{sec:SetupSheet}

\subsection{Lecture, Lecturer, Semester}

Setting up the exercise sheet is just a matter of calling a small number of commands
before the beginning of your document, in the preamble. See the template given in
Sec.~\ref{sec:quickstart}. These lines could be for example:
\begin{pkgverbatim}
\begin{verbatim}
\UebungLecture{Microstructures of molten cheese.}
\UebungProf{Prof. Zebigboss}
\UebungSemester{HS 2999}

\UebungsblattNumber{1}
\end{verbatim}
\end{pkgverbatim}

\pkgcmddoc[1]{UebungLecture}{This command sets the title of your lecture to the given
  argument. The lecture title is displayed in the main exercise header.}

\pkgcmddoc[1]{UebungProf}{Use this command to set the professor or lecturer of the course
  to the given argument.}

\pkgcmddoc[1]{UebungLecturer}{This command is an exact alias of \pkgcmd[]{UebungProf}.}

\pkgcmddoc[1]{UebungSemester}{This command sets the semester that will be displayed in the
  header.}

\pkgcmddoc[1]{UebungsblattNumber}{This sets the exercise sheet number to the given
  argument. The exercise sheet number usually starts at 1, and increases every week as
  more exercise sheets are distributed.}


\begin{pkgtip}
  These commands should be called in the preamble, but they just internally expand to an
  internal macro definition. So technically they can be called whenever you want. Just
  call them before calling any other macro that actually uses those values,
  e.g.~\pkgcmd{MakeUebungHeader}. Calling such a macro a second time with a different
  value overrides the previous value.
\end{pkgtip}

\subsection{Exercise Sheet Header}
\label{sec:Header}

The page header is generated automatically by the package, however you should call the
command \pkgcmd[]{MakeUebungHeader} explicitely at the beginning of the document.

\pkgcmddoc{MakeUebungHeader}{Draws the main header of the
  exercise sheet, in three parts, with ETH logo, centered title, and professor/semester
  displayed on the right. And a horizontal line under those.
}

The header automatically displays the right title, according to whether the exercise sheet
without the solutions or with the solutions is displayed, respectively printing ``Series''
or ``Solutions''. The appropriate titles are also automatically displayed in German when
the \texttt{german} package option is provided.

\begin{pkgtip}
  If you want to display some other string,
  like ``Exercise Sheet'', this title can be customized using
  commands~\pkgcmd{UebungsblattTitleSeries} and~\pkgcmd{UebungsblattTitleSolutions}. The
  font can also be changed, use~\pkgcmd{UebungsblattTitleFont}.

  The header itself is highly customizable, see Sec.~\ref{sec:CustomHeader}.
\end{pkgtip}


\subsection{Exercise Sheet Language: German or English}
\label{sec:Language}

\begin{pkgverbatim}
\begin{verbatim}
  \usepackage[deutsch]{ethuebung} % German version, "Uebungsblatt"
\end{verbatim}
\end{pkgverbatim}
will provide you the German version of the exercise sheet. Simply adding the \texttt{sol}
package option will provide you the ``Musterl\"osung'':
\begin{pkgverbatim}
\begin{verbatim}
  \usepackage[deutsch,sol]{ethuebung} % German verion, "Musterloesung"
\end{verbatim}
\end{pkgverbatim}

\begin{pkgtip}
  This package option does nothing else than redefining (re-customizing) the sheet title
  for exercises and for solutions, the exercise label using the
  commands~\pkgcmd{UebungsblattTitleSeries} etc. documented in
  section~\ref{sec:CustomizeSheet}. It also automatically includes the \LaTeX{}
  \texttt{babel} package with the \texttt{[german]} option.
\end{pkgtip}




\section{Exercises}

\subsection{\pkgenv[]{exenumerate} environment}

\pkgenvdoc{exenumerate}{This environment provides a \texttt{enumerate}-like environment,
  with labels (a), (b), ... by default, with which you can split an exercise into several
  parts. Use \cmdname{item} for each part, as for \texttt{itemize} and
  \texttt{enumerate}.}

Such \pkgenv[]{exenumerate} environments can be nested up to two levels (by default), and
the second level will be numbered (by default) (i), (ii), ... .

These environments may be broken and resumed, and their numbering will be automatically
resumed correctly and reset for each exercise. This is useful to add comments or to
introduce new concepts between different parts of an exercise.

For example:
\begin{pkgverbatim}
\verbatiminput{verbatim_exenumerate_resume.tex}%
\end{pkgverbatim}

Last but not least, you can refer to different parts of the exercise with \LaTeX's usual
\cmdname{label\pkgbraces{1}} and \cmdname{ref\pkgbraces{1}}
commands, as for example:
\begin{pkgverbatim}
\verbatiminput{verbatim_exenumerate_labelref.tex}%
\end{pkgverbatim}

\begin{pkgtip}
  See section~\ref{sec:CustomExenumerate} for commands available to customize the
  \pkgenv[]{exenumerate} environment, in particular~\pkgcmd[]{UebungLabelEnum}.
\end{pkgtip}


\subsection{Hints}
\label{sec:Hints}

Hints can be introduced with the \pkgcmd[]{hint} and \pkgcmd[]{hints} commands.

\pkgcmddoc[1]{hint}{Displays some text meant as a hint to the student with a label
  ``Hint''. A special font is used (e.g. small and italic)}

\pkgcmddoc[1]{hints}{Same as \pkgcmd[]{hint}, except uses the label ``Hints''. Use this
  when several hints are given at once.}

For example:
\begin{pkgverbatim}
\begin{verbatim}
  \hint{Remember that a unitary $U$ satisfies
      $UU^\dagger=U^\dagger U=\mathbb{I}$.}
\end{verbatim}
\end{pkgverbatim}
or, if there are several hints,
\begin{pkgverbatim}
\begin{verbatim}
  \hints{Remember that a unitary $U$ satisfies
      $UU^\dagger=U^\dagger U=\mathbb{I}$.

      Also, a rotation $R$ satisfies $RR^T=R^T R=\mathbb{I}$.
  }
\end{verbatim}
\end{pkgverbatim}

\begin{pkgtip}
  You can customize the appearance of the hint text, as well as the label used for hints
  with the~\pkgcmd{UebungHinweisLabel}, \pkgcmd{UebungHinweiseLabel}, and the
  \pkgcmd{UebungHinweisFont} commands.
\end{pkgtip}

\subsection{Splitting exercises into `Sub-Exercises'}
\label{sec:subexercises}

You can split exercises into sub-exercises, in the same spirit as when in a regular
\LaTeX{} article you split \cmdname{section}'s into \cmdname{subsection}'s.

\pkgcmddoc[1]{subuebung}{Define a sub-exercise, the title of which will be the argument
  given. This will number the sub-exercise automatically.}

\pkgcmddoc[1]{subexercise}{Exactly the same as \pkgcmd[]{subuebung}}


The following example:
\begin{pkgverbatim}
\begin{verbatim}
\exercise{Quantization of the Electromagnetic Field.}
In this exercise, we will learn to quantize the electromagnetic field.

\subexercise{Classical Case.}
First, here are some questions about classical E-M fields...

...

\subexercise{Quantum Case.}
Now we will quantize the E-M field...

...

\end{verbatim}
\end{pkgverbatim}
will appear as:
\begin{pkgverbatim}[0mm]
  {\bf Exercise N.} \hspace*{2mm} {\em\bfseries Quantization of the Electromagnetic Field.}

  In this exercise ....

  ...

  {\bf N.1}\hspace*{1mm} {\em\bfseries Classical Case.}\hspace*{3mm} First, here are ...

  ...

  {\bf N.2}\hspace*{1mm} {\em\bfseries Quantum Case.}\hspace*{3mm} Now we ...

  ...
  
\end{pkgverbatim}

\begin{pkgtip}
  Leaving an extra (blank) newline between \cmdname{subexercise} and the sub-exercise text
  will produce the sub-exercise text on a new line.
\end{pkgtip}

\begin{pkgtip}
  Of course, \pkgcmd[]{subexercise} is customizable, too. See section~\ref{sec:CustomTexts}.
\end{pkgtip}

\subsection{Note About Figures}

notes about figures?



\section{Solutions}
\label{sec:Solutions}

\subsection{Solutions are hidden or shown for the proper sheet version.}

You should write up the solutions for an exercise immedately after the exercise, or
between exercise parts, using
a~\pkgenv[]{loesung} environment. When the sheet is compiled in ``exercise sheet'' mode
(the default), then the solutions are simply ignored and not displayed. However, when the
package option {\tt sol} is provided, or if the command~\pkgcmd{UebungMakeSolutionsSheet}
is called, then the solutions are displayed with a ``Solution'' label, and (by default) in
a smaller font to make it visually clear that it is the solution to the exercise.

Formatting of the solutions takes care, too, of numbering the equations differently
(i.e. (S.1), (S.2), etc.) so equation numbering does not collide with the equation text
and all equations have the same labels. Equations are also guaranteed to have the same
labels between the exercise and solution versions of the sheet.

The~\pkgenv{loesung} environment may appear anywhere in the exercise, and may be
repeated. You may have, for example, one general solution at the end of the exercise, or
multiple solutions after each exercise part or sub-exercise.

\pkgenvdoc{loesung}{Solution to an exercise. The content of this environment is by default
  hidden, unless in `solution sheet' mode (package option {\tt sol}, or with the
  command~\pkgcmd{UebungMakeSolutionsSheet}). If in solution sheet mode, then the contents
  is formatted using a smaller font (by default) and is preceeded by the label
  ``Solution'' (or ``L\"osung'' if the sheet is in german, with the {\tt german} package
  option, see Sec.\ref{sec:Language}).

  Equations numbered within this environment obey a separate counter and their labels are
  preceeded by a letter ``S'' (resp ``L'' in German), i.e. (S.1), (S.2), ... (resp. (L.1),
  (L.2), ...), such that it is guaranteed that equation numbering stays consistent between
  solution sheet mode and exercise sheet mode.}

\pkgenvdoc{solution}{Exactly the same as the \pkgenv[]{loesung} environment.}

\begin{pkgnotice}
  Both commands~\pkgenv[]{loesung} and~\pkgenv[]{solution} display their label in the same
  language, which is the language of the exercise sheet. This defaults to English but may
  be set to German with the {\tt german} package option (Sec.~\ref{sec:Language}).
\end{pkgnotice}




\section{Customization}
\label{sec:CustomizeSheet}

\subsection{Customizing the Header}
\label{sec:CustomHeader}

\subsection{Customizing the \pkgenv[]{exenumerate} environment}
\label{sec:CustomExenumerate}

\subsection{Customizing the Exercise Sheet Texts and Labels}
\label{sec:CustomTexts}

title (ex/sol)

ex label and `composition' (``Exercise'', ``Question'' and/or ``Exercise N.'', ``Question N.'')

subexercise label and `composition'

\section{Some Internals}
\label{sec:Internals}


\section{Commands Reference}
\label{sec:AllCommands}

\pkgcmdindex{UebungLecture}{Sets the title of your lecture.}

\pkgcmdindex{UebungProf}{Sets the name of the professor or lecturer of the course.}

\pkgcmdindex{UebungLecturer}{Same as \pkgcmd{UebungProf}.}

\pkgcmdindex{UebungSemester}{Sets the course semester.}

\pkgcmdindex{UebungsblattNumber}{Sets the current exercise sheet number.}

\pkgcmdindex{UebungsblattTitleSeries}{Set the title for the exercise sheet (when
not in solutions mode).}

\pkgcmdindex{UebungsblattTitleSolutions}{Set the title for the solutions sheet (only when
  in solutions mode, i.e. with \texttt{[sol]} package option or with
  \pkgcmd{UebungMakeSolutionsSheet}).}



\end{document}

%%% Local Variables: 
%%% mode: latex
%%% TeX-master: t
%%% End: 
