\documentclass[11pt,a4paper]{article}

\usepackage{docnote}
\usepackage[T1]{fontenc}
\usepackage{framed}
\usepackage{verbatim}
\usepackage{docethuebung}


\title{Package {\fontseries{m}\selectfont ethuebung} for ETH Exercise Sheets}
\author{Philippe Faist, \texttt{pfaist@ethz.ch}}
\date\today

\begin{document}
\maketitle

\inlinetoc

\section{Quick Start Guide and Simple Usage.}
\label{sec:quickstart}

Copy the style file \texttt{ethuebung.sty} into the same directory as your exercise sheet
\LaTeX{} file. No other files are needed (no logo images etc. are needed as they are
packaged with the style file). Start off with the following minimal template.

\begin{pkgverbatim}%
\verbatiminput{verbatim_minimal_template.tex}%
\end{pkgverbatim}

If you want the {\bfseries German} version of the template, simply pass the
\texttt{[german]} option to the \texttt{\textbackslash usepackage} directive (see also
Sec.~\ref{sec:Language}),
\begin{pkgverbatim}
\begin{verbatim}
  \usepackage[deutsch]{ethuebung} % German version, "Uebungsblatt"
  %\usepackage[deutsch,sol]{ethuebung} % German verion, "Musterloesung"
\end{verbatim}
\end{pkgverbatim}

Obviously you'll have to fill in the title of your lecture, the lecturer and the semester
accordingly using the \pkgcmd{UebungLecture}, \pkgcmd{UebungProf} and
\pkgcmd{UebungSemester} commands. Use the \pkgcmd{UebungsblattNumber} command to specify
the number of the exercise sheet (usually increases by one every week). Don't forget the
\pkgcmd{MakeUebungHeader} command at the beginning of the document to actually draw the header.

Write the solutions to each exercise inline, using a \pkgenv{loesung} or \pkgenv{solution}
environment. When you compile the code above, then the solutions will not appear.

If you want to generate a solutions sheet with the solutions displayed (eg. for the TA's
or for the students after they handed in), then simply change the two lines:
\begin{pkgverbatim}
\begin{verbatim}
\usepackage{ethuebung} % comment this and uncomment next line for solutions
%\usepackage[sol]{ethuebung} % uncomment for solutions
\end{verbatim}
\end{pkgverbatim}
to:
\begin{pkgverbatim}
\begin{verbatim}
%\usepackage{ethuebung} % comment this and uncomment next line for solutions
\usepackage[sol]{ethuebung} % uncomment for solutions
\end{verbatim}
\end{pkgverbatim}

Recompile the sheet, and the solutions will be displayed.


\section{What This Package Does}

This package provides a unified way of typing exercises for ETH Zurich. While you type in
logically all aspects of your exercise using provided \LaTeX{} macros (title, text, hints,
solution, etc.), it is rendered according to some standard style (yet remaining highly
customizable), and provides different versions of the sheet for distributing to students
(without the solutions), or for TA's (with solutions).

\subsection{Exercise Sheet Header}
\label{sec:Header}

The page header is generated automatically by the package, however you should call the
command \pkgcmd[]{MakeUebungHeader} explicitely at the beginning of the document.

\pkgcmddoc{MakeUebungHeader}{Draws the main header of the
  exercise sheet, in three parts, with ETH logo, centered title, and professor/semester
  displayed on the right. And a horizontal line under those.
}

The header automatically displays the right title, according to whether the exercise sheet
without the solutions or with the solutions is displayed, respectively printing ``Series''
or ``Solutions''. The appropriate titles are also automatically displayed in German when
the \texttt{german} package option is provided.

\begin{pkgtip}
  If you want to display some other string,
  like ``Exercise Sheet'', this title can be customized using
  commands~\pkgcmd{UebungsblattTitleSeries} and~\pkgcmd{UebungsblattTitleSolutions}. The
  font can also be changed, use~\pkgcmd{UebungsblattTitleFont}.

  The header itself is highly customizable, see Sec.~\ref{sec:CustomHeader}.
\end{pkgtip}


\subsection{Exercise Sheet Language: German or English}
\label{sec:Language}

\begin{pkgverbatim}
\begin{verbatim}
  \usepackage[deutsch]{ethuebung} % German version, "Uebungsblatt"
\end{verbatim}
\end{pkgverbatim}
will provide you the German version of the exercise sheet. Simply adding the \texttt{sol}
package option will provide you the ``Musterl\"osung'':
\begin{pkgverbatim}
\begin{verbatim}
  \usepackage[deutsch,sol]{ethuebung} % German verion, "Musterloesung"
\end{verbatim}
\end{pkgverbatim}

\begin{pkgtip}
  This package option does nothing else than redefining (re-customizing) the sheet title
  for exercises and for solutions, the exercise label using the
  commands~\pkgcmd{UebungsblattTitleSeries} etc. documented in
  section~\ref{sec:CustomizeSheet}. It also automatically includes the \LaTeX{}
  \texttt{babel} package with the \texttt{[german]} option.
\end{pkgtip}



ex numbering

both ex sheet/solution sheet

  - eqns in solution numbered by themselves
  - can attach pdfs for solution


\section{Setting Up The Exercise Sheet}
\label{sec:SetupSheet}

Setting up the exercise sheet is just a matter of calling a small number of commands
before the beginning of your document, in the preamble. See the template given in
Sec.~\ref{sec:quickstart}. These lines could be for example:
\begin{pkgverbatim}
\begin{verbatim}
\UebungLecture{Microstructures of molten cheese.}
\UebungProf{Prof. Zebigboss}
\UebungSemester{HS 2999}

\UebungsblattNumber{1}
\end{verbatim}
\end{pkgverbatim}

\begin{pkgtip}
These commands should be called in the preamble, but they just internally expand to an
internal macro definition. So technically they can be called whenever you want. Just call
them before calling any other macro that actually uses those values,
e.g.~\pkgcmd{MakeUebungHeader}.
\end{pkgtip}

\pkgcmddoc[1]{UebungLecture}{This command sets the title of your lecture to the given
  argument. The lecture title is displayed in the main exercise header.}

\pkgcmddoc[1]{UebungProf}{Use this command to set the professor or lecturer of the course
  to the given argument.}

\pkgcmddoc[1]{UebungSemester}{This command sets the semester that will be displayed in the
  header.}

\pkgcmddoc[1]{UebungsblattNumber}{This sets the exercise sheet number to the given
  argument. The exercise sheet number usually starts at 1, and increases every week as
  more exercise sheets are distributed.}

\section{Exercises}

\subsection{\pkgenv[]{exenumerate} environment}

\pkgenvdoc{exenumerate}{This environment provides a \texttt{enumerate}-like environment,
  with labels (a), (b), ... by default, with which you can split an exercise into several
  parts. Use \texttt{\textbackslash item} for each part, like for \texttt{itemize} and
  \texttt{enumerate}.}

Such \pkgenv[]{exenumerate} environments can be nested up to two levels (by default), and
the second level will be numbered (by default) (i), (ii), ... .

These environments may be broken and resumed, and their numbering will be automatically
resumed correctly and reset for each exercise. This is useful to add comments or to
introduce new concepts between different parts of an exercise.

For example:
\begin{pkgverbatim}
\verbatiminput{verbatim_exenumerate_resume.tex}%
\end{pkgverbatim}

Last but not least, you can refer to different parts of the exercise with \LaTeX's usual
\texttt{\textbackslash label\pkgbraces{1}} and \texttt{\textbackslash ref\pkgbraces{1}}
commands, as for example:
\begin{pkgverbatim}
\verbatiminput{verbatim_exenumerate_labelref.tex}%
\end{pkgverbatim}


\subsection{Hints}
\label{sec:Hints}

\subsection{Splitting exercises into `Sub-Exercises'}
\label{sec:subexercises}

\subsection{Note About Figures}

notes about figures?



\section{Solutions}
\label{sec:Solutions}

\pkgenvdoc{loesung}{Solution to an exercise.................................... The
  language of the label displayed is that of the sheet, and not necessarily German.}

\pkgenvdoc{solution}{This is an exact alias of the \pkgenv[]{loesung} environment. The
  language of the label displayed is that of the sheet, and not necessarily English.}


\section{Customization}
\label{sec:CustomizeSheet}

\subsection{Customizing the Header}
\label{sec:CustomHeader}



\section{Some Internals}
\label{sec:Internals}


\section{Commands Reference}
\label{sec:AllCommands}

\pkgcmdindex{UebungsblattTitleSeries}{Set the title for the exercise sheet (when
not in solutions mode).}
\pkgcmdindex{UebungsblattTitleSolutions}{Set the title for the solutions sheet (only when
  in solutions mode, i.e. with \texttt{[sol]} package option or with
  \pkgcmd{UebungMakeSolutionsSheet}).}



\end{document}

%%% Local Variables: 
%%% mode: latex
%%% TeX-master: t
%%% End: 
