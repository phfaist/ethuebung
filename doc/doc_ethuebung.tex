\documentclass[11pt,a4paper]{article}

\usepackage{docnote}
\usepackage[T1]{fontenc}
\usepackage{framed}
\usepackage{verbatim}


\title{Package {\fontseries{m}\selectfont ethuebung} for ETH Exercise Sheets}
\author{Philippe Faist, \texttt{pfaist@ethz.ch}}
\date\today

\begin{document}
\maketitle

\newcounter{pkgdef}

\newcommand{\pkgcmd}[2][$^\text{(page~\pageref{pkgdef:\thepkgcmdname})}$]{\def\thepkgcmdname{#2}\texttt{\textbackslash{#2}}#1}
\newcommand{\pkgenv}[1]{\texttt{\textbackslash begin\{{#1}\}$\ldots$\textbackslash end\{{#1}\}}}


\newcounter{pkgbracestmp}
\newcommand{\pkgbracesdo}{%
  \ifnum\value{pkgbracestmp}=0%
    \message{is-zero.}%
  \else%
    \{$\ldots$\}%
    \addtocounter{pkgbracestmp}{-1}%
    \message{is \thepkgbracestmp}\pkgbracesdo%
  \fi%
}
\newcommand{\pkgbraces}[1]{\setcounter{pkgbracestmp}{#1}\pkgbracesdo}

\newcommand{\pkgobjectdefinition}[3]{%
  \vspace*{0.5cm}%
  \refstepcounter{pkgdef}\expandafter\label{pkgdef:#1}%
  \noindent\framebox{\hspace*{0.1\textwidth}\begin{minipage}{0.8\textwidth}\vspace*{1mm}%
      \noindent%
      #2%
      \vspace*{1mm}%
    \end{minipage}\hspace*{0.09\textwidth}%
  }%
  \vspace*{0.5cm}

}
\newcommand{\pkgcmddoc}[3][0]{\pkgobjectdefinition{#2}{%
    \hspace*{-1cm}\texttt{\textbf{\textbackslash #2\pkgbraces{#1}}}\hspace*{0.7cm}%
    #3%
  }%
}
\newcommand{\pkgenvdoc}[3][0]{\pkgobjectdefinition{#2}{%
    \hspace*{-1cm}\texttt{\textbf{\textbackslash begin\{#2\}\pkgbraces{#1} $\ldots$ \textbackslash end\{#2\}}}\\%
    #3%
  }%
}

\definecolor{shadecolor}{rgb}{0.95, 0.95, 0.95}
\newenvironment{pkgverbatim}{% begin defs
  \begin{oframed}%
}{% end defs
  \end{oframed}%
}





\section{Quick Start Guide and Simple Usage.}

Copy the style file \texttt{ethuebung.sty} into the same directory as your exercise sheet
\LaTeX{} file. No other files are needed (no logo images etc. are needed as they are
packaged with the style file). Start off with the following minimal template.

\begin{pkgverbatim}%
  \verbatiminput{verbatim_minimal_template.tex}%
\end{pkgverbatim}

Obviously you'll have to fill in the title of your lecture, the lecturer and the semester
accordingly using the \pkgcmd{UebungLecture}, \pkgcmd{UebungProf} and
\pkgcmd{UebungSemester} commands. Use the \pkgcmd{UebungsblattNumber} command to specify
the number of the exercise sheet (usually increases by one every week). Don't forget the
\pkgcmd{MakeUebungHeader} command at the beginning of the document to actually draw the header.

Write the solutions to each exercise inline, using a \pkgenv{loesung} or \pkgenv{solution}
environment. When you compile the code above, then the solutions will not appear.

If you want to generate a solutions sheet with the solutions displayed (eg. for the TA's
or for the students after they handed in), then simply change the two lines:
\begin{verbatim}
  \usepackage{ethuebung} % comment this and uncomment next line for solutions
  %\usepackage[sol]{ethuebung} % uncomment for solutions
\end{verbatim}
to:
\begin{verbatim}
  %\usepackage{ethuebung} % comment this and uncomment next line for solutions
  \usepackage[sol]{ethuebung} % uncomment for solutions
\end{verbatim}

Recompile the sheet, and the solutions will be displayed.


\section{What This Package Does}

This package provides a unified way of typing exercises for ETH Zurich. While you type in
logically all aspects of your exercise using provided \LaTeX{} macros (title, text, hints,
solution, etc.), it is rendered according to some predefied standard, and provides
different versions of the sheet for distributing to students (without the solutions), or
for TA's (with solutions).

The page header is generated automatically by the package, however the command
\pkgcmd[]{MakeUebungHeader} must be called explicitely at the beginning of the document.


\pkgcmddoc{MakeUebungHeader}{Draws the main header of the
  exercise sheet, in three parts, with ETH logo, centered title, and professor/semester
  displayed on the right. And a horizontal line under those.
}


header

ex numbering

both ex sheet/solution sheet

  - eqns in solution numbered by themselves
  - can attach pdfs for solution


\section{Setting Up The Exercise Sheet}

\pkgcmddoc[1]{UebungLecture}{This command sets the title of your lecture to the given
  argument. The lecture title is displayed in the main exercise header. Call this command
  once in your preamble.}

\pkgcmddoc[1]{UebungProf}{Use this command to set the professor or lecturer of the course
  to the given argument. Call this command once in your preamble.}

\pkgcmddoc[1]{UebungSemester}{This command sets the semester that will be displayed in the
  header. Call this command once in your preamble.}

\pkgcmddoc[1]{UebungsblattNumber}{This sets the exercise sheet number to the given
  argument. The exercise sheet number usually starts at 1, and increases every week as
  more exercise sheets are distributed. Call this command once in your preamble.}

\section{Exercises}

\subsection{\pkgenv[]{exenumerate} environment}

\pkgenvdoc{exenumerate}{This environment provides a \texttt{enumerate}-like environment,
  with labels (a), (b), ... by default, with which you can split an exercise into several
  parts. Use \texttt{\textbackslash item} for each part, like for \texttt{itemize} and
  \texttt{enumerate}.}

Such \pkgenv[]{exenumerate} environments can be nested up to two levels (by default), and
the second level will be numbered (by default) (i), (ii), ... .

These environments may be broken and resumed, and their numbering will be automatically
resumed correctly and reset for each exercise. This is useful to add comments or to
introduce new concepts between different parts of an exercise.

For example:
\begin{pkgverbatim}\begin{verbatim}
  Consider the setting in which one applies a positive voltage between the source and the
  gate leeds. Answer the following questions.
  \begin{exenumerate}
  \item Question 1
  \item Question 2
  \end{exenumerate}

  Now, consider setting a {\em negative} voltage instead.
  \begin{exenumerate}
  \item % This item will automatically be labelled (c).
    Recalculate the quantity blah blah for this setting.
  \end{exenumerate}
\end{verbatim}\end{pkgverbatim}

Last but not least, you can refer to different parts of the exercise with \LaTeX's usual
\texttt{\textbackslash label\pkgbraces{1}} and \texttt{\textbackslash ref\pkgbraces{1}}
commands, as for example:
\begin{pkgverbatim}\begin{verbatim}
  \begin{exenumerate}
  \item % This is item (a)
    \label{expart:FirstQuestion}
    Prove the existance of god.
  \item % This is item (b)
    Argue that question~\ref{expart:FirstQuestion} is quite difficult.
    % this will display "Argue that question (a) ..."
  \end{exenumerate}
\end{verbatim}\end{pkgverbatim}


hints

subexercises

notes about figures?

\section{Solutions}


\section{Customization}


\section{Internals}




\end{document}

%%% Local Variables: 
%%% mode: latex
%%% TeX-master: t
%%% End: 
