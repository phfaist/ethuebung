\documentclass[11pt,a4paper]{article}

\usepackage{docnote}
\usepackage[T1]{fontenc}
\usepackage{framed}
\usepackage{verbatim}
\usepackage{docethuebung}
\usepackage{url}


\title{Package {\fontseries{m}\selectfont ethuebung} for ETH Exercise Sheets}
\author{Philippe Faist, \texttt{pfaist@ethz.ch}}
\date\today

\begin{document}
\maketitle

\renewcommand{\abstractname}{}
\setlength{\noteabstracttextwidth}{0.95\textwidth}
\renewcommand{\noteabstracttextfont}{\small}
\begin{abstract}
This package provides a unified way of typing exercises for ETH Zurich. While you type in
logically all aspects of your exercise using provided \LaTeX{} macros (title, text, hints,
solution, etc.), it is rendered according to some standard style (yet remaining highly
customizable), and provides different versions of the sheet for distributing to students
(without the solutions), or for TA's (with solutions).
\end{abstract}

% TABLE OF CONTENTS
{\small
\inlinetoc
}

\section{Getting Started}

\subsection{Installation}

The simplest way of installing the \texttt{ethuebung} package is to copy the
\texttt{ethuebung.sty} file at the same location as your exercise sheet \LaTeX{} file. This
requires of course making several copies of the file if needed.
A cleaner installation is to place the \texttt{ethuebung.sty} file somewhere in your
\texttt{\$TEXINPUTS} environment path.

Note that no other file is needed. The ETH logo is embedded into the style file.


\subsection{Minimal Template}
\label{sec:minimaltemplate}

Here is a minimal template for an exercise sheet.

\begin{pkgverbatim}%
\verbatiminput{verbatim_minimal_template.tex}%
\end{pkgverbatim}


\subsection{What This Package Does}

This package provides a unified way of typing exercises for ETH Zurich. While you type in
logically all aspects of your exercise using provided \LaTeX{} macros (title, text, hints,
solution, etc.), it is rendered according to some standard style (yet remaining highly
customizable), and provides different versions of the sheet for distributing to students
(without the solutions), or for TA's (with solutions).

.................................

ex numbering

both ex sheet/solution sheet

  - eqns in solution numbered by themselves

  - can attach pdfs for solution


\section{Setting Up The Exercise Sheet}
\label{sec:SetupSheet}

\subsection{Lecture, Lecturer, Semester}

Setting up the exercise sheet is just a matter of calling a small number of commands
before the beginning of your document, in the preamble. See the template given in
Sec.~\ref{sec:minimaltemplate}. These lines could be for example:
\begin{pkgverbatim}
\begin{verbatim}
\UebungLecture{Microstructures of molten cheese.}
\UebungProf{Prof. Zebigboss}
\UebungSemester{HS 2999}

\UebungsblattNumber{1}
\end{verbatim}
\end{pkgverbatim}

\pkgcmddoc[\pkgbraces{1}]{UebungLecture}{This command sets the title of your lecture to the given
  argument. The lecture title is displayed in the main exercise header.}

\pkgcmddoc[\pkgbraces{1}]{UebungProf}{Use this command to set the professor or lecturer of the course
  to the given argument.}

\pkgcmddoc[\pkgbraces{1}]{UebungLecturer}{This command is an exact alias of \pkgcmd[]{UebungProf}.}

\pkgcmddoc[\pkgbraces{1}]{UebungSemester}{This command sets the semester that will be displayed in the
  header.}

\pkgcmddoc[\pkgbraces{1}]{UebungsblattNumber}{This sets the exercise sheet number to the given
  argument. The exercise sheet number usually starts at 1, and increases every week as
  more exercise sheets are distributed.

  The macro \pkgcmd[]{theUebungsblattNumber} is defined to expand to the given exercise
  sheet number.
}

\begin{pkgtip}
  These commands should be called in the preamble, but they just internally expand to an
  internal macro definition. So technically they can be called whenever you want. Just
  call them before calling any other macro that actually uses those values,
  e.g.~\pkgcmd{MakeUebungHeader}. Calling such a macro a second time with a different
  value overrides the previous value.
\end{pkgtip}


\pkgcmddoc{theUebungsblattNumber}{This macro expands to the current exercise sheet
  number. Set this macro by calling \pkgcmd[]{UebungsblattNumber}.

  You may use this macro anywhere in your document.
}




\subsection{Exercise Sheet Header}
\label{sec:Header}

The page header is generated automatically by the package, however you should call the
command \pkgcmd[]{MakeUebungHeader} explicitely at the beginning of the document.

\pkgcmddoc{MakeUebungHeader}{Draws the main header of the
  exercise sheet, in three parts, with ETH logo, centered title, and professor/semester
  displayed on the right. And a horizontal line under those.
}

The header automatically displays the right title, according to whether the exercise sheet
without the solutions or with the solutions is displayed, respectively printing ``Series''
or ``Solutions''. The appropriate titles are also automatically displayed in German when
the \texttt{german} package option is provided.

\begin{pkgtip}
  If you want to display some other string,
  like ``Exercise Sheet'', this title can be customized using
  commands~\pkgcmd{UebungsblattTitleSeries} and~\pkgcmd{UebungsblattTitleSolutions}. The
  font can also be changed, use~\pkgcmd{UebungsblattTitleFont}.

  The header itself is highly customizable, see Sec.~\ref{sec:CustomHeader}.
\end{pkgtip}


\subsection{Exercise Sheet Language: German or English}
\label{sec:Language}

\begin{pkgverbatim}
\begin{verbatim}
  \usepackage[deutsch]{ethuebung} % German version, "Uebungsblatt"
\end{verbatim}
\end{pkgverbatim}
will provide you the German version of the exercise sheet. Simply adding the \texttt{sol}
package option will provide you the ``Musterl\"osung'':
\begin{pkgverbatim}
\begin{verbatim}
  \usepackage[deutsch,sol]{ethuebung} % German verion, "Musterloesung"
\end{verbatim}
\end{pkgverbatim}

\begin{pkgtip}
  This package option does nothing else than redefining (re-customizing) the sheet title
  for exercises and for solutions, the exercise label using the
  commands~\pkgcmd{UebungsblattTitleSeries} etc. documented in
  section~\ref{sec:CustomizeSheet}. It also automatically includes the \LaTeX{}
  \texttt{babel} package with the \texttt{[german]} option.
\end{pkgtip}




\section{Exercises}

\subsection{\pkgcmd[]{uebung}: a new exercise}

Use~\pkgcmd[]{uebung} or~\pkgcmd[]{exercise} to introduce a new exercise, and specify a
title for your exercise.

\pkgcmddoc[\{$\langle$Exercise Title$\rangle$\}]{uebung}{Similar to a \LaTeX{}
  \cmdname{section} command, this command starts the definition of a new exercise. The
  exercises are automatically numbered. An adequate label is displayed with the current
  exercise number, and the exercise title is printed in bold italic font (by default).
}

The exercise is internally implemented as a \LaTeX{} \cmdname{paragraph}. The numbering is
taken care of by an internal counter (\texttt{uebcounter}).

The label and title font of the exercise can be highly customized by using or redefining
for example the commands~\pkgcmd{UebungLabel}, \pkgcmd{UebungExTitleFont},
\pkgcmd{theuebcounter} etc.

\pkgcmddoc[\pkgbraces1]{exercise}{Exactly the same as \pkgcmd[]{uebung}.}

\begin{pkgnotice}
  Commands~\pkgcmd[]{uebung} and~\pkgcmd[]{exercise} produce exactly the same output, in
  the same language, which is the language of the sheet. By default, the language is
  English, but it can be changed to German by specifying the \texttt{[deutsch]} package
  option.

  The label can also independently be changed, see~\pkgcmd{UebungLabel}.
\end{pkgnotice}



\subsection{The \pkgenv[]{exenumerate} environment}

\pkgenvdoc{exenumerate}{This environment provides a \texttt{enumerate}-like environment,
  with labels (a), (b), ... by default, with which you can split an exercise into several
  parts. Use \cmdname{item} for each part, as for \texttt{itemize} and
  \texttt{enumerate}.}

Such \pkgenv[]{exenumerate} environments can be nested up to two levels (by default), and
the second level will be numbered (by default) (i), (ii), ... .

These environments may be broken and resumed, and their numbering will be automatically
resumed correctly and reset for each exercise. This is useful to add comments or to
introduce new concepts between different parts of an exercise.

For example:
\begin{pkgverbatim}
\verbatiminput{verbatim_exenumerate_resume.tex}%
\end{pkgverbatim}

Last but not least, you can refer to different parts of the exercise with \LaTeX's usual
\cmdname{label\pkgbraces{1}} and \cmdname{ref\pkgbraces{1}}
commands, as for example:
\begin{pkgverbatim}
\verbatiminput{verbatim_exenumerate_labelref.tex}%
\end{pkgverbatim}

You may change the default labelling, (a), (b), ..., by specifying your label format as
\cmdname{begin\{exenumerate\}[format]}, for example:
\begin{pkgverbatim}
\begin{verbatim}
\begin{exenumerate}[A)]
\item This is A)
\item This is B)
\end{exenumerate}
\end{verbatim}
\end{pkgverbatim}
The syntax is the one used by the \pkgenv[]{enumerate} environment (in the
\texttt{enumerate}\footnote{\url{http://mirrors.ctan.org/macros/latex/required/tools/enumerate.pdf}}
package, or in the
\texttt{enumitem}\footnote{\url{http://mirrors.ctan.org/macros/latex/contrib/enumitem/enumitem.pdf}}
package with \texttt{shortlabels} options).

\begin{pkgtip}
  Internally the package \texttt{enumitem} is used, with option \texttt{shortlabels}. This
  allows the use as described above of the (old) \texttt{enumerate} syntax, as well as the
  new (but unfortunately more cryptic and verbose) \texttt{enumitem} syntax,
  \texttt{[label=(\textbackslash roman*)]}.

  See section~\ref{sec:CustomExenumerate} for commands available to customize the
  \pkgenv[]{exenumerate} environment, in particular~\pkgcmd[]{UebungLabelEnum}.
\end{pkgtip}


................. cross-refs .....................


\subsection{Hints}
\label{sec:Hints}

Hints can be introduced with the \pkgcmd[]{hint} and \pkgcmd[]{hints} commands.

\pkgcmddoc[\pkgbraces{1}]{hint}{Displays some text meant as a hint to the student with a label
  ``Hint''. A special font is used (e.g. small and italic)}

\pkgcmddoc[\pkgbraces{1}]{hints}{Same as \pkgcmd[]{hint}, except uses the label ``Hints''. Use this
  when several hints are given at once.}

For example:
\begin{pkgverbatim}
\begin{verbatim}
  \hint{Remember that a unitary $U$ satisfies
      $UU^\dagger=U^\dagger U=\mathbb{I}$.}
\end{verbatim}
\end{pkgverbatim}
or, if there are several hints,
\begin{pkgverbatim}
\begin{verbatim}
  \hints{Remember that a unitary $U$ satisfies
      $UU^\dagger=U^\dagger U=\mathbb{I}$.

      Also, a rotation $R$ satisfies $RR^T=R^T R=\mathbb{I}$.
  }
\end{verbatim}
\end{pkgverbatim}

\begin{pkgtip}
  You can customize the appearance of the hint text, as well as the label used for hints
  with the~\pkgcmd{UebungHinweisLabel}, \pkgcmd{UebungHinweiseLabel}, and the
  \pkgcmd{UebungHinweisFont} commands.
\end{pkgtip}

\pkgcmddoc[\pkgbraces1]{hinweis}{This is exactly the same as~\pkgcmd[]{hint}.}

\pkgcmddoc[\pkgbraces1]{hinweise}{This is exactly the same as~\pkgcmd[]{hints}.}

\begin{pkgnotice}
  Both~\pkgcmd[]{hinweis} and~\pkgcmd[]{hint} produce the same output in the same
  language, which is the language of the sheet (English by default, or German if the
  \texttt{deutsch} package option was given).
\end{pkgnotice}


\subsection{Splitting exercises into `Sub-Exercises'}
\label{sec:subexercises}

You can split exercises into sub-exercises, in the same spirit as when in a regular
\LaTeX{} article you split \cmdname{section}'s into \cmdname{subsection}'s.

\pkgcmddoc[\pkgbraces{1}]{subuebung}{Define a sub-exercise, the title of which will be the argument
  given. This will number the sub-exercise automatically.}

\pkgcmddoc[\pkgbraces{1}]{subexercise}{Exactly the same as \pkgcmd[]{subuebung}}


The following example:
\begin{pkgverbatim}
\begin{verbatim}
\exercise{Quantization of the Electromagnetic Field.}
In this exercise, we will learn to quantize the electromagnetic field.

\subexercise{Classical Case.}
First, here are some questions about classical E-M fields...

...

\subexercise{Quantum Case.}
Now we will quantize the E-M field...

...

\end{verbatim}
\end{pkgverbatim}
will appear as:
\begin{pkgverbatim}[0mm]
  {\bf Exercise N.} \hspace*{2mm} {\em\bfseries Quantization of the Electromagnetic Field.}

  In this exercise ....

  ...

  {\bf N.1}\hspace*{1mm} {\em\bfseries Classical Case.}\hspace*{3mm} First, here are ...

  ...

  {\bf N.2}\hspace*{1mm} {\em\bfseries Quantum Case.}\hspace*{3mm} Now we ...

  ...
  
\end{pkgverbatim}

\begin{pkgtip}
  Leaving an extra (blank) newline between \cmdname{subexercise} and the sub-exercise text
  will produce the sub-exercise text on a new line.
\end{pkgtip}

\begin{pkgtip}
  Of course, \pkgcmd[]{subexercise} is customizable, too. See section~\ref{sec:CustomTexts}.
\end{pkgtip}

\subsection{Note About Figures}

note about figures .......?
wrapfigure?


\section{Solutions}
\label{sec:Solutions}

\subsection{Inline Solutions for Solutions Sheet: \pkgenv[]{loesung}}

You should write up the solutions for an exercise immedately after the exercise, or
between exercise parts, using
a~\pkgenv[]{loesung} environment. When the sheet is compiled in ``exercise sheet'' mode
(the default), then the solutions are simply ignored and not displayed. However, when the
package option {\tt sol} is provided, or if the command~\pkgcmd{UebungMakeSolutionsSheet}
has been called, then whenever the environment \pkgenv[]{loesung} is encountered, a label
``Solution'' is printed, followed by the contents of the environment. By default, the
solution text is printed in a smaller font to make it visually clear that it is the
solution to the exercise.

Formatting of the solutions takes care, too, of numbering the equations differently
(i.e. (S.1), (S.2), etc.) so equation numbering does not collide with the equation
text. Equations are also guaranteed to have the same
labels between the exercise and solution versions of the sheet.

The~\pkgenv{loesung} environment may appear anywhere in the exercise, and may be
repeated. You may have, for example, one general solution at the end of the exercise, or
multiple solutions after each exercise part or sub-exercise.

\pkgenvdoc{loesung}{Provides solution text to an exercise. The content of this environment
  is by default hidden, unless in `solution sheet' mode (package option {\tt sol}, or with
  the command~\pkgcmd{UebungMakeSolutionsSheet}). If in solution sheet mode, then the
  contents is formatted using a smaller font (by default) and is preceeded by the label
  ``Solution'' (or ``L\"osung'' if the sheet is in german, with the {\tt german} package
  option, see Sec.\ref{sec:Language}).

  Equations numbered within this environment obey a separate counter and their labels are
  preceeded by a letter ``S'' (resp. ``L'' in German), i.e. (S.1), (S.2), ... (resp. (L.1),
  (L.2), ...), such that it is guaranteed that equation numbering stays consistent between
  solution sheet mode and exercise sheet mode.}

\pkgenvdoc{solution}{Exactly the same as the \pkgenv[]{loesung} environment.}

\begin{pkgnotice}
  Both commands~\pkgenv[]{loesung} and~\pkgenv[]{solution} display their label in the same
  language, which is the language of the exercise sheet. This defaults to English but may
  be set to German with the {\tt german} package option (Sec.~\ref{sec:Language}).
\end{pkgnotice}


\pkgcmddoc{UebungMakeSolutionsSheet}{This command has exactly the same effect as providing
  the \texttt{sol} package option. It switches the sheet to ``solutions'' mode, giving it
  a ``Solutions'' title, and displaying all contents provided in~\pkgenv[]{loesung}
  environments and~\pkgcmd{pdfsolution} commands.}


\subsection{PDF Attachment as solution}

It is also possible to attach a PDF file with, for example, a scanned hand-written
solution. The pages of that file are included at the end of the solutions sheet, with a
reference to the page number at the point in the exercise where the solution is
referenced, and a title superimposed to the included PDF pages that specify which exercise
those pages refer to.

For example, if a solution is scanned as \texttt{scanned-solution.pdf}, then you can
simply write in your exercise:
\begin{pkgverbatim}
\begin{verbatim}
\exercise{A Nice Exercise}
Show that blah blah blah something cool.

\pdfsolution{scanned-solution.pdf}

\end{verbatim}
\end{pkgverbatim}

the \pkgcmd[]{pdfsolution} command internally expands to a \pkgenv[]{solution} environment
at the point where its called, inserting some text like ``The solution is provided on page
XYZ''. Then the pages of the specified PDF are appended at the end of the solutions sheet,
with on each page printed on top ``Solution to exercise NN''.

This works also as inline solutions to sub-exercises and to exercise parts inside an\\
\pkgenv{exenumerate} environment.

Obviously, the command expands to nothing if not in solutions mode, and no PDF pages are
included in that case.

The inclusion of the PDF is internally accomplished with the~\cmdname{includepdf} command
from the
\texttt{pdfpages}\footnote{\url{http://mirrors.ctan.org/macros/latex/contrib/pdfpages/pdfpages.pdf}}
package.

\pkgcmddoc[{[options]\{pdf file\}}]{pdfsolution}{Specify a PDF file to include which
  contain the solution to the current exercise. A reference to the page where the
  solutions will be inserted (at the end of the sheet) is inserted at the current location
  within a~\pkgenv{solution} environment. The included pdf pages are given a title on each
  page specifying which exercise they are the solution to. The \texttt{options} are any
  options that can be passed to~\cmdname{includepdf} of the~\texttt{pdfpages} package.}

\pkgcmddoc[{[options]\{pdf file\}}]{pdfloesung}{Exact same command as~\pkgcmd[]{pdfsolution}.}



\section{Exercise Sheet Style}
\label{sec:styles}

This package provides different exercise sheet \emph{styles}, i.e.\ ways the sheet
look. The default style should usually be sufficient; however you may prefer the
appearance provided by other styles.

\pkgcmddoc[\pkgbraces1]{UebungStyle}{Sets the style of the exercise sheet to the given
  named style. This can be for example ``\texttt{Default}'' or ``\texttt{PreviousITP}''.
}

The \texttt{Default} style changes nothing. It just provides the default style with all
the labels and fonts and definitions as defined by the base package implementation.

The \texttt{PreviousITP} style redefines the appearance of the sheet to look like the
\texttt{exercise.sty} style developed by Christoph Buchendorfer, with a larger title font,
bold non italic exercise titles, etc.

Check out section~\ref{sec:customstyles} for information on how to define custom styles.

\section{Customizing the Exercise Sheet Appearance}
\label{sec:CustomizeSheet}

In this section we will present some handy commands to change the way the exercise sheet
looks, or to change some defaults to some other values you would prefer.

All \cmdname{Uebung***\pkgbraces1} commands specified in this section internally expand to
an internal macro definition, and may be called several times if need be (but it shouldn't
need be...). Also, best practice is to call them in the preamble, but things will still
work if you call them anywhere else before you call any command that actually uses the
values you want to set.

\subsection{Customizing the Header}
\label{sec:CustomHeader}

\pkgcmddoc[\pkgbraces1]{UebungsblattTitleSeries}{Redefines the title of the exercise
  sheet whenever the sheet is compiled in ``exercise'' sheet mode. This setting is ignored
  if the sheet is compiled in ``solutions'' sheet mode.

  Default title is ``{Series}'' (``{\"Ubungsblatt}'' in German).
}

\pkgcmddoc[\pkgbraces1]{UebungsblattTitleSolutions}{Redefines the title of the exercise
  sheet whenever the sheet is compiled in ``solutions'' mode. This setting is ignored if
  the sheet is compiled in ``exercise sheet'' mode.

  Default title is ``{Solutions}'' (``{Musterl\"osung}'' in German).
}

\begin{pkgtip}
  Both commands~\pkgcmd[]{UebungsblattTitleSeries}
  and~\pkgcmd[]{UebungsblattTitleSolutions} can be specified in the same preamble. The
  relevant title depending on exercise or solutions mode is automatically selected.
\end{pkgtip}

\pkgcmddoc[\pkgbraces1]{UebungsblattTitleFont}{Redefines the commands to set the font for
  the main title. You may use the usual \LaTeX{} comands to manipulate fonts,
  \cmdname{bfseries}, \cmdname{large}, \cmdname{Large}, \cmdname{fontfamily},
  \cmdname{fontseries}, \cmdname{selectfont} etc.

  It is also possible to pass the name of a macro that expects an argument,
  e.g.~\cmdname{underline}, if nothing else follows that macro.

  The default font specification is ``\cmdname{large}\cmdname{bfseries}''.
}


\pkgcmddoc[\pkgbraces1]{UebungLogoFile}{Specify a graphic file with a logo to include
  instead of the ETH logo. The file must be an acceptable file for
  \cmdname{includegraphics}, e.g.\ PDF or PNG for \texttt{pdf} output or EPS for
  \texttt{dvi} output.
}

\pkgcmddoc[\pkgbraces1]{UebungTitleCenterVSpacing}{Specify some extra spacing that will
  lift the central title a bit higher vertically with respect to the logo and the
  lecturer/semester. Set something here if you have e.g.\ a long lecture name which starts
  overlapping with the logo.

  By default, no spacing is added (``0mm'').
}


\subsection{Customizing the Exercise Labels and Fonts}
\label{sec:CustomExenumerate}
\label{sec:CustomTexts}

\pkgcmddoc[\pkgbraces1]{UebungHinweisFont}{Specify the font commands to use to set up the
  font for the main text produced by the~\pkgcmd{hint} and~\pkgcmd{hints} commands.

  This command has the same syntax as the \pkgcmd{UebungsblattTitleFont} command.

  Default font is ``\cmdname{small}\cmdname{em}''.
}

\pkgcmddoc[\pkgbraces1]{UebungExTitleFont}{Specify the font used when displaying the title
  of an exercise, i.e.\ the text passed as argument to the~\pkgcmd{exercise} command.

  This command has the same syntax as the \pkgcmd{UebungsblattTitleFont} command.

  Default font is ``\cmdname{bfseries}\cmdname{em}''.
}

\pkgcmddoc[\pkgbraces1]{UebungSubExTitleFont}{Specify the font used when displaying the
  title of a sub-exercise, i.e.\ the text passed as argument to the~\pkgcmd{subexercise}
  command.

  This command has the same syntax as the \pkgcmd{UebungsblattTitleFont} command.

  Default font is ``\cmdname{bfseries}\cmdname{em}''.
}

\pkgcmddoc[\pkgbraces1]{UebungLabel}{Specify the text to display to label an
  exercise. This is typically ``Exercise'' or ``Question''.

  Default value is ``{Exercise}''. See also \pkgcmd{uebTheUebungLabel}.
}

\pkgcmddoc[\pkgbraces1]{UebungSubLabel}{Specify the text to display to label a
  sub-exercise.

  The default value is empty. See also~\pkgcmd{uebTheUebungSubLabel}.
}

\pkgcmddoc[\pkgbraces1]{UebungLabelEnum}{Specify the label that will be used for each
  exercise part (e.g. (a), (b), ...) produced by a~\pkgenv{exenumerate} environment. Here
  you must specify the label with the format for
  the \texttt{enumitem} package, i.e.\ for example use any string containing one of the
  \cmdname{roman*}, \cmdname{Roman*}, \cmdname{alph*}, \cmdname{Alph*}, or
  \cmdname{arabic*} commands.

  The default label definition is ``\texttt{(\cmdname{alph*})}''.
}

\pkgcmddoc[\pkgbraces1]{UebungLabelEnumSub}{Specify the label that will be used for each
  nested exercise part, i.e.\ any \pkgenv{exenumerate} environment nested within
  another~\pkgenv{exenumerate} environment. The default is (i), (ii), ... As
  for~\pkgcmd[]{UebungLabelEnum}, the format has to be conform to the \texttt{enumitem}
  package.

  The default label definition is ``\texttt{(\cmdname{roman*})}''.
}

\pkgcmddoc[\pkgbraces1]{UebungHinweisLabel}{Specify what text to display to introduce a
  hint produced by~\pkgcmd{hint}. You may for example use a colon (`:') instead of a
  period (`.') if you prefer.

  Default text is ``{Hint.}'' (``{Hinweis.}'' in German).
}

\pkgcmddoc[\pkgbraces1]{UebungHinweiseLabel}{Specify what text to display to introduce
  hints produced by~\pkgcmd{hints}. You may for example use a colon (`:') instead of a
  period (`.') if you prefer.

  Default text is ``{Hints.}'' (``{Hinweise.}'' in German).
}

\begin{pkgnotice}
  Note that the same text will be displayed by the~\pkgcmd{hint} and~\pkgcmd{hinweis}
  commands regardless of the sheet language setting. That text is the one specified
  to~\pkgcmd[]{UebungHinweisLabel}. The sheet
  language setting only changes the default value for the hint label. The same applies
  to the~\pkgcmd{hints} and~\pkgcmd{hinweise} commands.

  Actually, internally, when the the~\texttt{deutsch} package option is given, a call
  to~\pkgcmd[]{UebungHinweisLabel} is made to set the German version of the label, to
  replace the initial English version.
\end{pkgnotice}


\subsection{Customizing the Solutions Labels and Fonts}


\pkgcmddoc[\pkgbraces1]{UebungLoesungFont}{Specify the font to use for the inline
  solutions environment \pkgenv{loesung}. The font is a sequence of \LaTeX{} font commands
  such as \cmdname{bfseries}, \cmdname{sfshape} etc.

  This command has the same syntax as \pkgcmd{UebungsblattTitleFont}. However, here you
  may not specify a command that will take an argument such as \cmdname{underline}.
}

\pkgcmddoc[\pkgbraces1]{UebungSolLabel}{Specify the text to display as a label for an
  inline solution block. This is the label that will be displayed as title of the
  \cmdname{(sub)paragraph} which is induced by a \pkgenv{loesung} environment.

  The default is ``Solution.'' (or in German, ``L\"osung.'').
}

\pkgcmddoc[\pkgbraces1]{UebungSolEquationLabel}{Specify a label that should be used to
  identify equations that are inside a solutions environment. This text is prepended to
  the equation number.

  The default is ``S.'' (or in German, ``L.'').
}

\pkgcmddoc[\pkgbraces1]{UebungAttachedSolutionTitleTop}{For pages with solutions included
  from an external PDF file with \pkgcmd{pdfloesung}, you may specify here the spacing
  between the top of the page and the title which is automatically produced on those
  pages.

  The default value is ``\texttt{1.3\textbackslash baselineskip}''.
}

\pkgcmddoc[\pkgbraces1]{UebungAttachedSolutionTitleFont}{For pages with solutions included
  from an external PDF file with \pkgcmd{pdfloesung}, specify here the font commands to
  use to display the title which is automatically produced on those pages.

  The syntax is the same as \pkgcmd{UebungsblattTitleFont}.

  The default value is ``\cmdname{bfseries}\cmdname{small}\cmdname{underline}''.
}

\pkgcmddoc[\pkgbraces1]{UebungAttachedSolutionTitle}{For pages with solutions included
  from an external PDF file with \pkgcmd{pdfloesung}, specify here the title to
  automatically display on each of those pages.

  You may use the special
  \cmdname{uebattachedsolutiontheexercisenumber} command to refer to the exercise for
  which the current page is the solution.

  The default value is ``\texttt{Solution to\\
  Exercise\textasciitilde\cmdname{uebattachedsolutiontheexercisenumber}.}'', or, if the
  sheet is in German, ``\texttt{L\textbackslash"osung zu der\\
  \textbackslash"Ubung\textasciitilde\cmdname{uebattachedsolutiontheexercisenumber}.}''.
}

\pkgcmddoc[\pkgbraces1]{UebungTextAttachedSolution}{For solutions included from an
  external PDF file with \pkgcmd{pdfloesung}, use this command to specify which text
  should be displayed at the point where \pkgcmd[]{pdfloesung} is called, to refer to an
  attached solution.

  You may use the special \cmdname{uebthepageattached} command to refer to the page on
  which the solution is attached.

  The default value is ``\texttt{The solution to this exercise is attached on
  page~\cmdname{uebthepageattached}.}'', or, if the sheet is in German,
  ``\texttt{Die L\textbackslash"osung dieser \textbackslash"Ubung finden Sie im Anhang auf
    der Seite\textasciitilde\cmdname{uebthepageattached}.}''.
}




\subsection{Customizable ``Composed'' Commands}

Some compositions can be redefined in order to change
the sheet appearance.

\pkgcmddoc{theuebcounter}{This command expands to the current exercise number, formatted
  in the way we want to display it in the sheet. It is a standard \LaTeX{} counter value
  like \cmdname{thesection}.

  The default is simply ``\texttt{\cmdname{arabic}\{uebcounter\}}''.

  You may redefine this command to display the exercise number as you wish. For example,
%  you could use the format ``$\langle$Sheet \#$\rangle$.$\langle$ N $\rangle$'' with the
  redefinition
%\begin{verbatim}
%  \renewcommand{\theuebcounter}{\theUebungsblattNumber.\arabic{uebcounter}}
%\end{verbatim}
...................
}

\pkgcmddoc{uebTheUebungLabel}{This command expands to the label that should be used as a
  title for the \cmdname{(sub)paragraph} generated by \pkgcmd{uebung}.
  You may redefine this command to some custom value, which will probably refer to the
  current exercise number, \cmdname{theuebcounter}.

  The default is\\
  ``\texttt{\textbackslash ueb@maybespaceafter\{\textbackslash
    ueb@TheUebungLabel\}\textbackslash theuebcounter.}''.
}

\pkgcmddoc{uebTheUebungSubLabel}{Same as \pkgcmd[]{uebTheUebungLabel}, except that it is
  used for sub-exercises (\pkgcmd[]{subuebung}/\pkgcmd{subexercise}).

  The relevant counter value is \cmdname{thesubuebcounter}, which should be defined
  normally to also display the main exercise number.

  The default is\\
  ``\texttt{\textbackslash ueb@maybespaceafter\{\textbackslash
    ueb@TheUebungSubLabel\}\textbackslash thesubuebcounter.}''.
}



%\newcommand{\ueb@displayheader}{%


ex label and `composition' (``Exercise'', ``Question'' and/or ``Exercise N.'', ``Question N.'')

subexercise label and `composition'


\paragraph{Some \LaTeX{} Utilities.} Some utilities are defined by this package,
which you may use when you redefine commands presented above.
All utilities are properly commented inside the source code itself, and only what I think
are the two most useful utilities are presented here.

Note that to use these
commands in a regular \LaTeX{} file and not a style file, you should enclose your
definitions between \cmdname{makeatletter} and \cmdname{makeatother}, since the command
names contain the ``\texttt{\char`@}'' character.

\pkgcmddoc[\pkgbraces1]{ueb@maybespace}{Expands to the content of the argument preceeded
  by a nonbreaking space (\texttt\textasciitilde), except if the contents contains no
  text, in which case this macro expands to nothing.
}

\pkgcmddoc[\pkgbraces1]{ueb@maybespaceafter}{Same as \pkgcmd[]{ueb@maybespace}, except
  that the space is inserted \emph{after} the content if the content is not empty.
}



\subsection{Defining Custom Styles}
\label{sec:customstyles}

A `style' is simply a collection of customization commands such as presented above,
combined together in a \LaTeX{} macro that is named
\cmdname{uebstyle@<StyleName>}. If you also provide the command named
\cmdname{uebstyle@<StyleName>@<Language>}, where \texttt{<Language>} is the
language of the sheet (i.e.\ ``English'' or ``Deutsch''), then it is called, too.

A (very simple) custom style could be defined as
\begin{pkgverbatim}
\begin{verbatim}
 \makeatletter
 \newcommand{\uebstyle@DummyStyle}{%
   \UebungLabelEnum{\alph*)}%
   \UebungLabelEnumSub{\roman*)}%
   \UebungLoesungFont{}%
   \UebungExTitleFont{\bfseries}%
 }
 \newcommand{\uebstyle@DummyStyle@Deutsch}{%
   \UebungLabel{Aufgabe}%
 }
 \makeatother
\end{verbatim}
\end{pkgverbatim}

This style could be set by calling the command
\begin{pkgverbatim}
\begin{verbatim}
  \UebungStyle{DummyStyle}
\end{verbatim}
\end{pkgverbatim}

The exercise sheet would then have exercise parts numbered ``a)'', ``b)'', etc., the
solutions font would be normal font and not a small font, and the exercise titles would be
bold instead of bold italic. Also, if the sheet happened to be in German, then the
exercise label would be ``Aufgabe X.'' instead of ``\"Ubung X.''.

\begin{pkgnotice}
  If you define custom styles, \emph{please} it would be nice to let me know, so that I
  can include them in future versions of \texttt{ethuebung}. Contact me at
  \texttt{pfaist@ethz.ch}.
\end{pkgnotice}






\subsection{Some Internals}
\label{sec:Internals}

The style sheet is thoroughly commented and should be pretty readable. It uses some
\LaTeX{} hacks, which I have tried to document properly.

Contact me if you need help decyphering the code, or if you have suggestions or
comments. Just send me an e-mail at \texttt{pfaist@ethz.ch}.



\section{Commands Reference}
\label{sec:AllCommands}

\pkgcmdindex{UebungLecture}{Sets the title of your lecture.}

\pkgcmdindex{UebungProf}{Sets the name of the professor or lecturer of the course.}

\pkgcmdindex{UebungLecturer}{Same as \pkgcmd{UebungProf}.}

\pkgcmdindex{UebungSemester}{Sets the course semester.}

\pkgcmdindex{UebungsblattNumber}{Sets the current exercise sheet number.}

\pkgcmdindex{UebungsblattTitleSeries}{Set the title for the exercise sheet (when
not in solutions mode).}

\pkgcmdindex{UebungsblattTitleSolutions}{Set the title for the solutions sheet (only when
  in solutions mode, i.e. with \texttt{[sol]} package option or with
  \pkgcmd{UebungMakeSolutionsSheet}).}


\section{Package Options Reference}
\label{sec:PackageOptions}

deutsch

sol

noenum

nogeom


\end{document}

%%% Local Variables: 
%%% mode: latex
%%% TeX-master: t
%%% End: 
