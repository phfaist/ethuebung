\documentclass[11pt,a4paper]{article}

\usepackage{docnote}


\title{Package {\sf ethuebung} for ETH Exercise Sheets}
\author{Philippe Faist, \texttt{pfaist@ethz.ch}}
\date\today


\begin{document}
\maketitle

\section{Quick Start Guide and Simple Usage.}

Copy the style file \texttt{ethuebung.sty} into the same directory as your exercise sheet
\LaTeX{} file. No other files are needed (no logo images etc. are needed as they are
packaged with the style file).

Start off with the minimal template
\begin{verbatim}
  \documentclass[11pt,a4paper]{article}
  
  \usepackage{ethuebung} % comment this and uncomment next line for solutions
  %\usepackage[sol]{ethuebung} % uncomment for solutions
  
  \UebungLecture{Microstructures of molten cheese.}
  \UebungProf{Prof. Zebigboss}
  \UebungSemester{HS 2999}
  
  \UebungsblattNumber{1}
  
  \begin{document}
  \MakeUebungHeader
  
  \exercise{Title of the exercise.}
  
  In this exercise, you will be asked to do some work.
  
  \begin{exenumerate}
  \item Solve the following equation.
    \begin{align}
      x^2 = 1\ .
    \end{align}
    \hint{There might be more than one solution.}
  
    \begin{solution}
    \end{solution}
  \end{exenumerate}

  \end{document}
\end{verbatim}

Obviously you'll have to fill in the title of your lecture, the lecturer and the semester
accordingly using the \pkgcmd{UebungLecture}, \pkgcmd{UebungProf} and
\pdgcmd{UebungSemester} commands. Use the \pkgcmd{UebungsblattNumber} command to specify
the number of the exercise sheet (usually increases by one every week).

\section{Basic Usage}


\end{document}

%%% Local Variables: 
%%% mode: latex
%%% TeX-master: t
%%% End: 
